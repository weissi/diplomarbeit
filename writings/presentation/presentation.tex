\documentclass[compress]{beamer}

\usepackage{beamerthemesplit}

% language
\usepackage[american]{babel}
\usepackage[utf8x]{inputenc}

\usepackage{ucs}
\usepackage{graphicx}
\usepackage{multirow}

% bib
\bibliographystyle{alphaurl}

% TikZ
\usepackage{tikz}

% beamer
\usetheme{Frankfurt}
\usecolortheme{default}
\usepackage{appendixnumberbeamer}

%
% SPECIAL IMPORTS
%
\usepackage{xparse}
\usepackage{boxedminipage}


%
% GENERAL
%

% environments
\NewDocumentEnvironment{JWboxed}{mm}%
  {\begin{figure}[ht]%
   \small%
   \begin{boxedminipage}{\linewidth}%
  }%
  {\end{boxedminipage}%
   \caption{#1}%
   \label{#2}%
   \end{figure}%
  }

% functionality
\newenvironment{JWfunc}[3]%
{\begin{JWboxed}{#2}{#3}%
 \begin{center}\textbf{{Functionality #1}}\end{center}%
}%
{\end{JWboxed}}

\newenvironment{JWfuncSteps}[0]{\begin{itemize}}{\end{itemize}}

\newcommand{\JWfuncSym}[2]{$\mathcal{F}^\mathrm{#1}_\mathrm{#2}$}

%protocol
\newenvironment{JWprotocol}[3]%
{\begin{JWboxed}{#2}{#3}%
 \begin{center}\textbf{{Protocol #1}}\end{center}%
}%
{\end{JWboxed}}

\newenvironment{JWprotoSteps}[0]{\begin{enumerate}}{\end{enumerate}}

\newcommand{\JWprotoPhase}[1]{\paragraph{#1}}

\newcommand{\JWprotoSym}[2]{$\Pi^\mathrm{#1}_\mathrm{#2}$}

%misc
\newcommand{\JWbinary}[1]{\texttt{#1}}
\newcommand{\JWtodo}[1]{\fbox{TODO: #1}}
\newcommand{\JWmsgTT}[2]{(\texttt{#1}, \texttt{#2})}
\newcommand{\JWmsgT}[1]{(\texttt{#1})}
\newcommand{\JWmsgTP}[2]{(\texttt{#1}, #2)}
\newcommand{\JWpath}[1]{\texttt{#1}}

\DeclareDocumentCommand\JWdef{mmg}{%
  {\label{def:#2}\emph{#1\IfNoValueF{#3}{#3}}%
  } (#2%
  \IfNoValueF{#3}{#3}%
  )%
}

\newenvironment{JWtodoBox}[0]%
  {\begin{boxedminipage}{\linewidth}TODO:\par}{\end{boxedminipage}}

%chapters
\makeatletter
\newcommand{\JWlone}{\@ifstar
  \JWloneNoTOC%
  \JWloneTOC%
}
\newcommand{\JWltwo}{\@ifstar
  \JWltwoNoTOC%
  \JWltwoTOC%
}
\newcommand{\JWlthree}{\@ifstar
  \JWlthreeNoTOC%
  \JWlthreeTOC%
}
\newcommand{\JWlfour}{\@ifstar
  \JWlfourNoTOC%
  \JWlfourTOC%
}
\newcommand{\JWlfive}{\@ifstar
  \JWlfiveNoTOC%
  \JWlfiveTOC%
}
\newcommand{\JWlsix}{\@ifstar
  \JWlsixNoTOC%
  \JWlsixTOC%
}

\newcommand{\JWloneTOC}[1]{\chapter{#1}}
\newcommand{\JWloneNoTOC}[1]{\chapter*{#1}}
\newcommand{\JWltwoTOC}[1]{\section{#1}}
\newcommand{\JWltwoNoTOC}[1]{\section*{#1}}
\newcommand{\JWlthreeTOC}[1]{\subsection{#1}}
\newcommand{\JWlthreeNoTOC}[1]{\subsection*{#1}}
\newcommand{\JWlfourTOC}[1]{\subsubsection{#1}}
\newcommand{\JWlfourNoTOC}[1]{\subsubsection*{#1}}
\newcommand{\JWlfiveTOC}[1]{\paragraph{#1}}
\newcommand{\JWlfiveNoTOC}[1]{\paragraph*{#1}}
\newcommand{\JWlsixTOC}[1]{\subparagraph{#1}}
\newcommand{\JWlsixNoTOC}[1]{\subparagraph*{#1}}

%links and names
\newcommand{\JWnamedlink}[2]{\href{#1}{#2}}
\newcommand{\JWurl}[1]{\href{#1}{#1}}

%code/commands
\newcommand{\JWcode}[1]{\texttt{#1}}
\newcommand{\JWcmd}[1]{\texttt{\$ #1}}
\newcommand{\JWhsExt}[1]{\texttt{#1}}


%lemmata/theorems
\newtheorem{thm}{Theorem}
\newtheorem{lem}[thm]{Lemma}


%
% SPECIALIZED SHORTCUTS
%
\newcommand{\JWprotoSymOPE}[0]{\JWprotoSym{}{OPE}}
\newcommand{\JWfuncSymOPE}[0]{\JWfuncSym{(k, n)}{OPE}}
\newcommand{\JWfuncSymOPEnp}[0]{\JWfuncSym{}{OPE}}
\newcommand{\JWfuncSymOAFE}[0]{\JWfuncSym{seq-ot}{OAFE}}
\newcommand{\JWfieldGeneral}[0]{\mathbb{F}_{2^k}}
\newcommand{\JWadv}[0]{$\mathcal{A}$}
\newcommand{\JWpOne}[0]{Goliath}
\newcommand{\JWpTwo}[0]{David}
\newcommand{\JWtoken}[0]{Token}
\newcommand{\JWnegl}[0]{\text{*) Except for a negligible probability.}}
\newcommand{\JWtitle}[0]{Efficient Secure Function Evaluation using Garbled %
Arithmetic Circuits and Untrusted Tamper--Proof Hardware}
\newcommand{\entspricht}{\stackrel{\scriptscriptstyle\wedge}{=}}

%tools ans versions
\newcommand{\JWThaskell}[0]{Haskell}
\newcommand{\JWTghc}[0]{GHC}
\newcommand{\JWTXghc}[0]{Glasgow Haskell Compiler}
\newcommand{\JWTLhaskell}[0]{\JWnamedlink{http://haskell.org}{\JWThaskell}}
\newcommand{\JWTLhaddock}[0]{\JWnamedlink{http://www.haskell.org/haddock/}%
                            {Haddock}}
\newcommand{\JWTLcabal}[0]{\JWnamedlink{http://www.haskell.org/cabal/}%
                            {Cabal}}
\newcommand{\JWTXLghc}[0]{\JWnamedlink{http://www.haskell.org/ghc/}{\JWTXghc}}
\newcommand{\JWTVghc}{7.6.1}
\newcommand{\JWTntl}[0]{NTL}
\newcommand{\JWTLntl}[0]{\JWnamedlink{http://www.shoup.net/ntl/}{\JWTntl}}
\newcommand{\JWTcpp}[0]{C++}
\newcommand{\JWTc}[0]{C}
\newcommand{\JWTgit}[0]{git}
\newcommand{\JWTquickcheck}[0]{QuickCheck}
\newcommand{\JWTLhunit}[0]{\JWnamedlink{http://hunit.sourceforge.net/}{HUnit}}
\newcommand{\JWTLmonadcryptorandom}[0]%
  {\JWnamedlink{http://hackage.haskell.org/package/monadcryptorandom}%
  {\texttt{monadcryptorandom}}}
\newcommand{\JWTLhaskellForMaths}[0]%
  {\JWnamedlink{http://hackage.haskell.org/package/HaskellForMaths}%
  {\texttt{HaskellForMaths}}}
\newcommand{\JWTLdrbg}[0]%
  {\JWnamedlink{http://hackage.haskell.org/package/DRBG}{\texttt{DRBG}}}
\newcommand{\JWTXprotobuf}{Google\TTra{} Protocol Buffers}
\newcommand{\JWTprotobuf}{Protocol Buffers}
\newcommand{\JWTXLprotobuf}[0]%
  {\JWnamedlink{http://code.google.com/apis/protocolbuffers/}{\JWTXprotobuf{}}}
\newcommand{\JWTLhackage}[0]%
  {\JWnamedlink{http://hackage.haskell.org/packages/hackage.html}{HackageDB}}


%binaries
\newcommand{\JWBpOne}[0]{\JWbinary{\JWpOne{}}}
\newcommand{\JWBpTwo}[0]{\JWbinary{\JWpTwo{}}}
\newcommand{\JWBtoken}[0]{\JWbinary{\JWtoken{}}}

%misc
\newcommand{\JWport}[1]{\texttt{#1}}
\def\TReg{\textsuperscript{\textregistered}}
\def\TCop{\textsuperscript{\textcopyright}}
\def\TTra{\textsuperscript{\texttrademark}}


\title{\JWtitle{}}

\author{Johannes Weiß}

\date{Januar 2012}

\begin{document}

\frame{\titlepage}

%
% OUTLINE
%
\section{Outline}

\frame {

  \frametitle{Outline}

  \begin{itemize}

    \item What is the Problem?

    \item OAFE

    \item Secure Arithmetic

    \item Demo

  \end{itemize}

}


%
% WHAT IS THE PROBLEM
%

\section{Problem}

% WHAT IS THE PROBLEM
\subsection{What is the Problem?}

\frame {

  \frametitle{What is the Problem?}

  \begin{itemize}

    \item \emph{Secure Function Evaluation} (SFE)\\
      \JWtodo{Kurze Erklärung was das ist}

    \item \emph{Oblivious Polynomial Evaluation} (OPE)\\
      \JWtodo{Kurze Erklärung was das ist}

  \end{itemize}

}

% OTHER APPROACHES
\subsection{Other Approaches}

\frame {

  \frametitle{Other Approaches}

  \begin{itemize}

    \item Garbled Arithmetic Circuits \cite{gac2012}

    \item \JWtodo{Andere hinschreiben}

  \end{itemize}

}

% INITIAL IDEA
\subsection{Initial Idea}

\frame {

  \frametitle{Initial Idea}

  Just implement the ideas of some papers...

}

% OPE
\subsection{OPE}

\frame {

  \frametitle{Oblivious Polynomial Evaluation (OPE)}

  \begin{align*}
    %
    y = f(x) = \sum_{i=0}^n a_ix^i
    %
  \end{align*}

  \begin{tikzpicture}[>=stealth]

    \node (OPE) at (8.5,0) {OPE};
    \draw (OPE) +(-1.5,-1.15) rectangle +(1.5,0.65);

    \draw [<-] (OPE) ++(-1.5,0.25) node [anchor=west] {} -- +(-0.75,0) node
    [anchor=east] {$\JWfieldGeneral^{n+1} \ni a$};

    \draw [<-] (OPE) ++(1.5,0.25) node [anchor=east] {} -- +(0.75,0) node
    [anchor=west] {$x \in \JWfieldGeneral$};

    \draw [->] (OPE) ++(1.5,-0.75) node [anchor=east] {$\sum_{i=0}^n a_ix^i$}
    -- +(0.75,0)
    node [anchor=west] {$y \in \JWfieldGeneral$};

  \end{tikzpicture}

  The first party chooses the polynomial and learns nothing, the second party
  chooses the node $x$ and learns $y = f(x)$.

}


%
% OAFE
%

\section{OAFE}

\frame {

  \frametitle{Oblivious Affine Function Evaluation (OAFE)}

  \JWtodo{Erklärung was OAFE \cite{davidgoliath} ist}

}


%
% SECURE ARITHMETIC
%

\section{Secure Arithmetic}

\subsection{Definition}

\frame {

  \frametitle{Secure Arithmetic}

  \JWtodo{Erklärung was Secure Arithmetic so leisten muss}

}

\frame {

  \frametitle{Secure Arithmetic}

  \begin{JWtodoBox}

    \begin{itemize}

      \item Double--Zeug

      \item Wegmann Carter MAC

    \end{itemize}

  \end{JWtodoBox}

}

% ADDITION
\subsection{Addition}

\frame {

  \frametitle{Addition}

  \JWtodo{Wie funktioniert $a + b$}

}

% MULTIPLICATION
\subsection{Multiplication}

\frame {

  \frametitle{Multiplication}

  \JWtodo{Wie funktioniert $a \cdot b$}

  \JWtodo{Hier die erste Double--Konstruktion erklären}

}

\frame {

  \frametitle{Problem}

  \JWtodo{Problem der Konstruktion erklären}

}

\frame {

  \frametitle{The Radicals Trick}

  \JWtodo{Radicals Trick erklären}

}


%
% EVALUATION
%

\section{Evaluation}

\frame {

  \frametitle{Evaluation}

  \JWtodo{Bilder und Evaluation}

}


%
% CONTRIBUTION
%

\section{Contribution}

\frame {

  \frametitle{Contribution}

  \begin{itemize}

    \item UC--secure

    \item linear time complexity

  \end{itemize}

}


%
% IMPLEMENTATION
%
\section{Implementation}

\subsection{Implementation}

\frame {

  \frametitle{Implementation}

  \JWtodo{Some information about the implementation}

}

\frame {

  \frametitle{Demonstration of the Implementation}

  DEMO

}


%
% APPENDIX
%

\appendix

\section{Bibliography}

\frame {

  \frametitle{Bibliography}

  \bibliography{bibliography}

}

\end{document}

% vim: set spell spelllang=en_us fileencoding=utf8 formatoptions=tcroql : %
