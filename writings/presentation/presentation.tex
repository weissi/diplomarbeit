\documentclass[compress]{beamer}

\usepackage{beamerthemesplit}

% language
\usepackage[american]{babel}
\usepackage[utf8x]{inputenc}

\usepackage{ucs}
\usepackage{graphicx}
\usepackage{multirow}

% bib
\bibliographystyle{alphaurl}

% TikZ
\usepackage{tikz}

% beamer
\usetheme{Frankfurt}
\usecolortheme{default}
\usepackage{appendixnumberbeamer}

%
% SPECIAL IMPORTS
%
\usepackage{xparse}
\usepackage{boxedminipage}


%
% GENERAL
%

% environments
\NewDocumentEnvironment{JWboxed}{mm}%
  {\begin{figure}[ht]%
   \small%
   \begin{boxedminipage}{\linewidth}%
  }%
  {\end{boxedminipage}%
   \caption{#1}%
   \label{#2}%
   \end{figure}%
  }

% functionality
\newenvironment{JWfunc}[3]%
{\begin{JWboxed}{#2}{#3}%
 \begin{center}\textbf{{Functionality #1}}\end{center}%
}%
{\end{JWboxed}}

\newenvironment{JWfuncSteps}[0]%
{\begin{itemize}}{\end{itemize}}

\newcommand{\JWfuncSym}[2]{$\mathcal{F}^\mathrm{#1}_\mathrm{#2}$}

%protocol
\newenvironment{JWprotocol}[3]%
{\begin{JWboxed}{#2}{#3}%
 \begin{center}\textbf{{Protocol #1}}\end{center}%
}%
{\end{JWboxed}}

\newenvironment{JWprotoSteps}[0]%
{\begin{enumerate}}{\end{enumerate}}

\newcommand{\JWprotoPhase}[1]{\paragraph{#1}}

\newcommand{\JWprotoSym}[2]{$\Pi^\mathrm{#1}_\mathrm{#2}$}

%misc
\newcommand{\JWtodo}[1]{\fbox{TODO: #1}}
\newcommand{\JWmsgTwo}[2]{(\texttt{#1}, \texttt{#2})}

%chapters
\newcommand{\JWlone}[1]{\chapter{#1}}
\newcommand{\JWltwo}[1]{\section{#1}}
\newcommand{\JWlthree}[1]{\subsection{#1}}
\newcommand{\JWlfour}[1]{\subsubsection{#1}}
\newcommand{\JWlfive}[1]{\paragraph{#1}}
\newcommand{\JWlsix}[1]{\subparagraph{#1}}


%
% SPECIALIZED SHORTCUTS
%
\newcommand{\JWprotoSymOPE}[0]{\JWprotoSym{}{OPE}}
\newcommand{\JWfuncSymOPE}[0]{\JWfuncSym{(q, k)}{OPE}}
\newcommand{\JWfieldGeneral}[0]{\mathbb{F}_q^k}
\newcommand{\JWpOne}[0]{Goliath}
\newcommand{\JWpTwo}[0]{David}


\title{\JWtitle{}}

\author{Johannes Weiß}

\date{Januar 2012}

\begin{document}

\frame{\titlepage}

%
% OUTLINE
%
\section{Outline}

\frame {

  \frametitle{Outline}

  \begin{itemize}

    \item What is the Problem?

    \item OAFE

    \item Secure Arithmetic

    \item Demo

  \end{itemize}

}


%
% WHAT IS THE PROBLEM
%

\section{Problem}

% WHAT IS THE PROBLEM
\subsection{What is the Problem?}

\frame {

  \frametitle{What is the Problem?}

  \begin{itemize}

    \item \emph{Secure Function Evaluation} (SFE)\\
      \JWtodo{Kurze Erklärung was das ist}

    \item \emph{Oblivious Polynomial Evaluation} (OPE)\\
      \JWtodo{Kurze Erklärung was das ist}

  \end{itemize}

}

% OTHER APPROACHES
\subsection{Other Approaches}

\frame {

  \frametitle{Other Approaches}

  \begin{itemize}

    \item Garbled Arithmetic Circuits \cite{gac2012}

    \item \JWtodo{Andere hinschreiben}

  \end{itemize}

  \begin{JWtodoBox}

    Problems:

    \begin{itemize}

      \item No Square \& Multiply

      \item Polynomial Time Complexity

    \end{itemize}

  \end{JWtodoBox}

}

% INITIAL IDEA
\subsection{Initial Idea}

\frame {

  \frametitle{Initial Idea}

  SFE on arbitrary arithmetic circuits.

  ... bt just implementing the ideas of some papers.

  Didn't work, why?

}

% OPE
\subsection{OPE}

\frame {

  \frametitle{Oblivious Polynomial Evaluation (OPE)}

  \begin{align*}
    %
    y = f(x) = \sum_{i=0}^n a_ix^i
    %
  \end{align*}

  \begin{tikzpicture}[>=stealth]

    \node (OPE) at (8.5,0) {OPE};
    \draw (OPE) +(-1.5,-1.15) rectangle +(1.5,0.65);

    \draw [<-] (OPE) ++(-1.5,0.25) node [anchor=west] {} -- +(-0.75,0) node
    [anchor=east] {$\JWfieldGeneral^{n+1} \ni a$};

    \draw [<-] (OPE) ++(1.5,0.25) node [anchor=east] {} -- +(0.75,0) node
    [anchor=west] {$x \in \JWfieldGeneral$};

    \draw [->] (OPE) ++(1.5,-0.75) node [anchor=east] {$\sum_{i=0}^n a_ix^i$}
    -- +(0.75,0)
    node [anchor=west] {$y \in \JWfieldGeneral$};

  \end{tikzpicture}

  The first party chooses the polynomial and learns nothing, the second party
  chooses the node $x$ and learns $y = f(x)$.

}


%
% OAFE
%

\section{OAFE}

\frame {

  \frametitle{Oblivious Affine Function Evaluation (OAFE)}

  \JWtodo{Erklärung was OAFE \cite{davidgoliath} ist}

}


%
% SECURE ARITHMETIC
%

\section{Secure Arithmetic}

\subsection{Definition}

\frame {

  \frametitle{Secure Arithmetic}

  \JWtodo{Erklärung was Secure Arithmetic so leisten muss}

}

\frame {

  \frametitle{Secure Arithmetic}

  \begin{JWtodoBox}

    \begin{itemize}

      \item Double--Zeug

      \item Wegmann Carter MAC

    \end{itemize}

  \end{JWtodoBox}

}

% ADDITION
\subsection{Addition}

\frame {

  \frametitle{Addition}

  \JWtodo{Wie funktioniert $a + b$}

}

% MULTIPLICATION
\subsection{Multiplication}

\frame {

  \frametitle{Multiplication}

  \JWtodo{Wie funktioniert $a \cdot b$}

  \JWtodo{Hier die erste Double--Konstruktion erklären}

}

\frame {

  \frametitle{Problem}

  \JWtodo{Problem der Konstruktion erklären}

}

\frame {

  \frametitle{The Radicals Trick}

  \JWtodo{Radicals Trick erklären}

}


%
% EVALUATION
%

\section{Evaluation}

\frame {

  \frametitle{Evaluation}

  \JWtodo{Bilder und Evaluation}

}


%
% CONTRIBUTION
%

\section{Contribution}

\frame {

  \frametitle{Contribution}

  \begin{itemize}

    \item UC--secure

    \item linear time complexity

  \end{itemize}

}


%
% IMPLEMENTATION
%
\section{Implementation}

\subsection{Implementation}

\frame {

  \frametitle{Implementation}

  \JWtodo{Some information about the implementation}

}

\frame {

  \frametitle{Demonstration of the Implementation}

  DEMO

}


%
% APPENDIX
%

\appendix

\section{Bibliography}

\frame {

  \frametitle{Bibliography}

  \bibliography{bibliography}

}

\end{document}

% vim: set spell spelllang=en_us fileencoding=utf8 formatoptions=tcroql : %
