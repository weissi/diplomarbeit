\JWlone{Evaluation}
\label{sec:evaluation}

\JWltwo{Test Setup}
\label{sec:test-setup}

\JWlthree{Test Maschines}
\label{sec:test-machines}

\JWlfour*{Test Maschine 1}

\emph{Dell Latitude D620} running \emph{Debian GNU/Linux} on kernel version
\emph{3.1.7}. The machine is powered by a \emph{Intel\TReg{} Core Duo T2400}
dual core processor at \emph{1.83 GHz}. It has \emph{2 GB} system memory.


\JWlfour*{Test Maschine 2}

\emph{Apple Mac mini} (\texttt{Macmini 3,1}) running \emph{MacOS X 10.7.5
(Lion)}. The machine is powered by a \emph{Intel\TReg{} Core 2 Duo} dual core
processor at \emph{2.26 GHz}. It has \emph{8 GB} system memory.

\JWlthree{Test Setup}

Both machines run the three programs (\JWBpOne{}, \JWBpTwo{}, \JWBtoken{})
simultaneously, the programs are compiled single--threaded. The programs
communicate to each other using local TCP/IP networking. Chapter
\ref{sec:communication-channels} has a in depth description about the
communication that takes part in the implementation that comes with this thesis.
The running time includes the DRAC building and transmitting, the OAFE
configuration and the successful evaluation of a random polynomial in the field
$\mathbb{F}_{2^{256}}$. The polynomial is evaluated using \emph{Horner's
rule}\cite{cormen01}.


\JWltwo{Running Time}
\label{sec:running-time}

Ordinary evaluation of a polynomial using \emph{Horner's rule} is in $\Theta(n)$
\cite{cormen01}. Despite the complex technique presented in this thesis the
running time increases nearly linear in the polynomial degree. The small
non--linear part comes from the large amount of memory the large degree
polynomials take to evaluate. Especially the marshalling (and unmarshalling) of
the large data--structure that is necessary before transmission via network is
quite memory intensive and gives the garbage collector a lot to do. This means
that polynomial evaluation is only slower than the common polynomial evaluation
by a constant factor. Figure \ref{fig:poly-deg-t} and table \ref{tab:poly-deg-t}
demonstrate that. An implementation whose goal is optimal performance could very
likely drop the non--linear factor entirely. Figure \ref{fig:poly-deg-t-small}
and table \ref{tab:poly-deg-t-small} motivate this because the non--linearity is
invisible for polynomials of smaller degree ($< 300$).

The difference in the running time is nearly equal between both Test Machines
(see figure \ref{fig:poly-deg-t}), this is because the marshalling/unmarshalling
and network transmission are the most relevant factors for the running time.

\begin{figure}[ht]
  \centering
  % Created by tikzDevice version 0.6.2-92-0ad2792 on 2012-12-13 23:56:52
% !TEX encoding = UTF-8 Unicode
\begin{tikzpicture}[x=1pt,y=1pt]
\definecolor[named]{fillColor}{rgb}{1.00,1.00,1.00}
\path[use as bounding box,fill=fillColor,fill opacity=0.00] (0,0) rectangle (433.62,505.89);
\begin{scope}
\path[clip] ( 49.20, 61.20) rectangle (408.42,456.69);
\definecolor[named]{drawColor}{rgb}{0.00,0.00,0.00}

\path[draw=drawColor,line width= 0.4pt,line join=round,line cap=round] ( 62.50, 78.42) --
	( 80.01, 81.23) --
	( 97.52, 84.03) --
	(115.02, 86.69) --
	(132.53, 89.18) --
	(150.03, 91.82) --
	(167.54, 94.60) --
	(185.05, 97.16) --
	(202.55, 99.90) --
	(220.06,102.64) --
	(237.56,105.35) --
	(255.07,108.07) --
	(272.57,110.86) --
	(290.08,113.43) --
	(307.59,116.38) --
	(325.09,119.07) --
	(342.60,122.01) --
	(360.10,124.80) --
	(377.61,127.88) --
	(395.12,130.62);

\path[draw=drawColor,line width= 0.4pt,line join=round,line cap=round] ( 62.50, 78.42) circle (  2.25);

\path[draw=drawColor,line width= 0.4pt,line join=round,line cap=round] ( 80.01, 81.23) circle (  2.25);

\path[draw=drawColor,line width= 0.4pt,line join=round,line cap=round] ( 97.52, 84.03) circle (  2.25);

\path[draw=drawColor,line width= 0.4pt,line join=round,line cap=round] (115.02, 86.69) circle (  2.25);

\path[draw=drawColor,line width= 0.4pt,line join=round,line cap=round] (132.53, 89.18) circle (  2.25);

\path[draw=drawColor,line width= 0.4pt,line join=round,line cap=round] (150.03, 91.82) circle (  2.25);

\path[draw=drawColor,line width= 0.4pt,line join=round,line cap=round] (167.54, 94.60) circle (  2.25);

\path[draw=drawColor,line width= 0.4pt,line join=round,line cap=round] (185.05, 97.16) circle (  2.25);

\path[draw=drawColor,line width= 0.4pt,line join=round,line cap=round] (202.55, 99.90) circle (  2.25);

\path[draw=drawColor,line width= 0.4pt,line join=round,line cap=round] (220.06,102.64) circle (  2.25);

\path[draw=drawColor,line width= 0.4pt,line join=round,line cap=round] (237.56,105.35) circle (  2.25);

\path[draw=drawColor,line width= 0.4pt,line join=round,line cap=round] (255.07,108.07) circle (  2.25);

\path[draw=drawColor,line width= 0.4pt,line join=round,line cap=round] (272.57,110.86) circle (  2.25);

\path[draw=drawColor,line width= 0.4pt,line join=round,line cap=round] (290.08,113.43) circle (  2.25);

\path[draw=drawColor,line width= 0.4pt,line join=round,line cap=round] (307.59,116.38) circle (  2.25);

\path[draw=drawColor,line width= 0.4pt,line join=round,line cap=round] (325.09,119.07) circle (  2.25);

\path[draw=drawColor,line width= 0.4pt,line join=round,line cap=round] (342.60,122.01) circle (  2.25);

\path[draw=drawColor,line width= 0.4pt,line join=round,line cap=round] (360.10,124.80) circle (  2.25);

\path[draw=drawColor,line width= 0.4pt,line join=round,line cap=round] (377.61,127.88) circle (  2.25);

\path[draw=drawColor,line width= 0.4pt,line join=round,line cap=round] (395.12,130.62) circle (  2.25);
\end{scope}
\begin{scope}
\path[clip] (  0.00,  0.00) rectangle (433.62,505.89);
\definecolor[named]{drawColor}{rgb}{0.00,0.00,0.00}

\path[draw=drawColor,line width= 0.4pt,line join=round,line cap=round] (115.02, 61.20) -- (395.12, 61.20);

\path[draw=drawColor,line width= 0.4pt,line join=round,line cap=round] (115.02, 61.20) -- (115.02, 55.20);

\path[draw=drawColor,line width= 0.4pt,line join=round,line cap=round] (185.05, 61.20) -- (185.05, 55.20);

\path[draw=drawColor,line width= 0.4pt,line join=round,line cap=round] (255.07, 61.20) -- (255.07, 55.20);

\path[draw=drawColor,line width= 0.4pt,line join=round,line cap=round] (325.09, 61.20) -- (325.09, 55.20);

\path[draw=drawColor,line width= 0.4pt,line join=round,line cap=round] (395.12, 61.20) -- (395.12, 55.20);

\node[text=drawColor,anchor=base,inner sep=0pt, outer sep=0pt, scale=  1.00] at (115.02, 39.60) {2000};

\node[text=drawColor,anchor=base,inner sep=0pt, outer sep=0pt, scale=  1.00] at (185.05, 39.60) {4000};

\node[text=drawColor,anchor=base,inner sep=0pt, outer sep=0pt, scale=  1.00] at (255.07, 39.60) {6000};

\node[text=drawColor,anchor=base,inner sep=0pt, outer sep=0pt, scale=  1.00] at (325.09, 39.60) {8000};

\node[text=drawColor,anchor=base,inner sep=0pt, outer sep=0pt, scale=  1.00] at (395.12, 39.60) {10000};

\path[draw=drawColor,line width= 0.4pt,line join=round,line cap=round] ( 49.20, 75.85) -- ( 49.20,442.04);

\path[draw=drawColor,line width= 0.4pt,line join=round,line cap=round] ( 49.20, 75.85) -- ( 43.20, 75.85);

\path[draw=drawColor,line width= 0.4pt,line join=round,line cap=round] ( 49.20,149.09) -- ( 43.20,149.09);

\path[draw=drawColor,line width= 0.4pt,line join=round,line cap=round] ( 49.20,222.33) -- ( 43.20,222.33);

\path[draw=drawColor,line width= 0.4pt,line join=round,line cap=round] ( 49.20,295.56) -- ( 43.20,295.56);

\path[draw=drawColor,line width= 0.4pt,line join=round,line cap=round] ( 49.20,368.80) -- ( 43.20,368.80);

\path[draw=drawColor,line width= 0.4pt,line join=round,line cap=round] ( 49.20,442.04) -- ( 43.20,442.04);

\node[text=drawColor,rotate= 90.00,anchor=base,inner sep=0pt, outer sep=0pt, scale=  1.00] at ( 34.80, 75.85) {0};

\node[text=drawColor,rotate= 90.00,anchor=base,inner sep=0pt, outer sep=0pt, scale=  1.00] at ( 34.80,149.09) {50};

\node[text=drawColor,rotate= 90.00,anchor=base,inner sep=0pt, outer sep=0pt, scale=  1.00] at ( 34.80,222.33) {100};

\node[text=drawColor,rotate= 90.00,anchor=base,inner sep=0pt, outer sep=0pt, scale=  1.00] at ( 34.80,295.56) {150};

\node[text=drawColor,rotate= 90.00,anchor=base,inner sep=0pt, outer sep=0pt, scale=  1.00] at ( 34.80,368.80) {200};

\node[text=drawColor,rotate= 90.00,anchor=base,inner sep=0pt, outer sep=0pt, scale=  1.00] at ( 34.80,442.04) {250};

\path[draw=drawColor,line width= 0.4pt,line join=round,line cap=round] ( 49.20, 61.20) --
	(408.42, 61.20) --
	(408.42,456.69) --
	( 49.20,456.69) --
	( 49.20, 61.20);
\end{scope}
\begin{scope}
\path[clip] (  0.00,  0.00) rectangle (433.62,505.89);
\definecolor[named]{drawColor}{rgb}{0.00,0.00,0.00}

\node[text=drawColor,anchor=base,inner sep=0pt, outer sep=0pt, scale=  1.00] at (228.81, 15.60) {Polynomial Degree};

\node[text=drawColor,rotate= 90.00,anchor=base,inner sep=0pt, outer sep=0pt, scale=  1.00] at ( 10.80,258.94) {Running Time [s]};
\end{scope}
\begin{scope}
\path[clip] ( 49.20, 61.20) rectangle (408.42,456.69);
\definecolor[named]{drawColor}{rgb}{0.00,0.00,0.00}

\path[draw=drawColor,line width= 0.4pt,dash pattern=on 4pt off 4pt ,line join=round,line cap=round] ( 62.50, 79.35) --
	( 80.01, 84.68) --
	( 97.52, 90.73) --
	(115.02, 98.08) --
	(132.53,106.17) --
	(150.03,113.29) --
	(167.54,122.69) --
	(185.05,134.33) --
	(202.55,145.06) --
	(220.06,157.07) --
	(237.56,168.86) --
	(255.07,181.74) --
	(272.57,200.62) --
	(290.08,213.09) --
	(307.59,228.52) --
	(325.09,247.93) --
	(342.60,263.12) --
	(360.10,285.49) --
	(377.61,305.55) --
	(395.12,328.11);

\path[draw=drawColor,line width= 0.4pt,line join=round,line cap=round] ( 62.50, 79.35) circle (  2.25);

\path[draw=drawColor,line width= 0.4pt,line join=round,line cap=round] ( 80.01, 84.68) circle (  2.25);

\path[draw=drawColor,line width= 0.4pt,line join=round,line cap=round] ( 97.52, 90.73) circle (  2.25);

\path[draw=drawColor,line width= 0.4pt,line join=round,line cap=round] (115.02, 98.08) circle (  2.25);

\path[draw=drawColor,line width= 0.4pt,line join=round,line cap=round] (132.53,106.17) circle (  2.25);

\path[draw=drawColor,line width= 0.4pt,line join=round,line cap=round] (150.03,113.29) circle (  2.25);

\path[draw=drawColor,line width= 0.4pt,line join=round,line cap=round] (167.54,122.69) circle (  2.25);

\path[draw=drawColor,line width= 0.4pt,line join=round,line cap=round] (185.05,134.33) circle (  2.25);

\path[draw=drawColor,line width= 0.4pt,line join=round,line cap=round] (202.55,145.06) circle (  2.25);

\path[draw=drawColor,line width= 0.4pt,line join=round,line cap=round] (220.06,157.07) circle (  2.25);

\path[draw=drawColor,line width= 0.4pt,line join=round,line cap=round] (237.56,168.86) circle (  2.25);

\path[draw=drawColor,line width= 0.4pt,line join=round,line cap=round] (255.07,181.74) circle (  2.25);

\path[draw=drawColor,line width= 0.4pt,line join=round,line cap=round] (272.57,200.62) circle (  2.25);

\path[draw=drawColor,line width= 0.4pt,line join=round,line cap=round] (290.08,213.09) circle (  2.25);

\path[draw=drawColor,line width= 0.4pt,line join=round,line cap=round] (307.59,228.52) circle (  2.25);

\path[draw=drawColor,line width= 0.4pt,line join=round,line cap=round] (325.09,247.93) circle (  2.25);

\path[draw=drawColor,line width= 0.4pt,line join=round,line cap=round] (342.60,263.12) circle (  2.25);

\path[draw=drawColor,line width= 0.4pt,line join=round,line cap=round] (360.10,285.49) circle (  2.25);

\path[draw=drawColor,line width= 0.4pt,line join=round,line cap=round] (377.61,305.55) circle (  2.25);

\path[draw=drawColor,line width= 0.4pt,line join=round,line cap=round] (395.12,328.11) circle (  2.25);

\path[draw=drawColor,line width= 0.4pt,dash pattern=on 1pt off 3pt ,line join=round,line cap=round] ( 62.50, 78.65) --
	( 80.01, 81.57) --
	( 97.52, 84.58) --
	(115.02, 87.65) --
	(132.53, 90.79) --
	(150.03, 93.74) --
	(167.54, 96.64) --
	(185.05, 99.65) --
	(202.55,102.66) --
	(220.06,105.75) --
	(237.56,108.80) --
	(255.07,112.23) --
	(272.57,115.52) --
	(290.08,118.78) --
	(307.59,122.14) --
	(325.09,125.20) --
	(342.60,128.92) --
	(360.10,132.36) --
	(377.61,136.27) --
	(395.12,139.64);

\path[draw=drawColor,line width= 0.4pt,line join=round,line cap=round] ( 60.51, 76.66) rectangle ( 64.50, 80.65);

\path[draw=drawColor,line width= 0.4pt,line join=round,line cap=round] ( 78.02, 79.58) rectangle ( 82.00, 83.56);

\path[draw=drawColor,line width= 0.4pt,line join=round,line cap=round] ( 95.52, 82.59) rectangle ( 99.51, 86.57);

\path[draw=drawColor,line width= 0.4pt,line join=round,line cap=round] (113.03, 85.66) rectangle (117.02, 89.64);

\path[draw=drawColor,line width= 0.4pt,line join=round,line cap=round] (130.53, 88.79) rectangle (134.52, 92.78);

\path[draw=drawColor,line width= 0.4pt,line join=round,line cap=round] (148.04, 91.75) rectangle (152.03, 95.73);

\path[draw=drawColor,line width= 0.4pt,line join=round,line cap=round] (165.55, 94.65) rectangle (169.53, 98.63);

\path[draw=drawColor,line width= 0.4pt,line join=round,line cap=round] (183.05, 97.65) rectangle (187.04,101.64);

\path[draw=drawColor,line width= 0.4pt,line join=round,line cap=round] (200.56,100.67) rectangle (204.55,104.65);

\path[draw=drawColor,line width= 0.4pt,line join=round,line cap=round] (218.06,103.76) rectangle (222.05,107.75);

\path[draw=drawColor,line width= 0.4pt,line join=round,line cap=round] (235.57,106.81) rectangle (239.56,110.80);

\path[draw=drawColor,line width= 0.4pt,line join=round,line cap=round] (253.07,110.23) rectangle (257.06,114.22);

\path[draw=drawColor,line width= 0.4pt,line join=round,line cap=round] (270.58,113.52) rectangle (274.57,117.51);

\path[draw=drawColor,line width= 0.4pt,line join=round,line cap=round] (288.09,116.79) rectangle (292.07,120.78);

\path[draw=drawColor,line width= 0.4pt,line join=round,line cap=round] (305.59,120.14) rectangle (309.58,124.13);

\path[draw=drawColor,line width= 0.4pt,line join=round,line cap=round] (323.10,123.21) rectangle (327.09,127.19);

\path[draw=drawColor,line width= 0.4pt,line join=round,line cap=round] (340.60,126.93) rectangle (344.59,130.91);

\path[draw=drawColor,line width= 0.4pt,line join=round,line cap=round] (358.11,130.37) rectangle (362.10,134.36);

\path[draw=drawColor,line width= 0.4pt,line join=round,line cap=round] (375.62,134.27) rectangle (379.60,138.26);

\path[draw=drawColor,line width= 0.4pt,line join=round,line cap=round] (393.12,137.65) rectangle (397.11,141.64);

\path[draw=drawColor,line width= 0.4pt,dash pattern=on 1pt off 3pt on 4pt off 3pt ,line join=round,line cap=round] ( 62.50, 79.07) --
	( 80.01, 83.86) --
	( 97.52, 90.24) --
	(115.02, 99.15) --
	(132.53,108.37) --
	(150.03,118.82) --
	(167.54,130.21) --
	(185.05,145.10) --
	(202.55,159.29) --
	(220.06,175.49) --
	(237.56,192.41) --
	(255.07,210.45) --
	(272.57,232.79) --
	(290.08,254.01) --
	(307.59,278.04) --
	(325.09,305.31) --
	(342.60,329.45) --
	(360.10,358.18) --
	(377.61,390.64) --
	(395.12,426.25);

\path[draw=drawColor,line width= 0.4pt,line join=round,line cap=round] ( 60.51, 77.07) rectangle ( 64.50, 81.06);

\path[draw=drawColor,line width= 0.4pt,line join=round,line cap=round] ( 78.02, 81.87) rectangle ( 82.00, 85.86);

\path[draw=drawColor,line width= 0.4pt,line join=round,line cap=round] ( 95.52, 88.25) rectangle ( 99.51, 92.23);

\path[draw=drawColor,line width= 0.4pt,line join=round,line cap=round] (113.03, 97.15) rectangle (117.02,101.14);

\path[draw=drawColor,line width= 0.4pt,line join=round,line cap=round] (130.53,106.38) rectangle (134.52,110.37);

\path[draw=drawColor,line width= 0.4pt,line join=round,line cap=round] (148.04,116.83) rectangle (152.03,120.82);

\path[draw=drawColor,line width= 0.4pt,line join=round,line cap=round] (165.55,128.22) rectangle (169.53,132.21);

\path[draw=drawColor,line width= 0.4pt,line join=round,line cap=round] (183.05,143.10) rectangle (187.04,147.09);

\path[draw=drawColor,line width= 0.4pt,line join=round,line cap=round] (200.56,157.29) rectangle (204.55,161.28);

\path[draw=drawColor,line width= 0.4pt,line join=round,line cap=round] (218.06,173.50) rectangle (222.05,177.48);

\path[draw=drawColor,line width= 0.4pt,line join=round,line cap=round] (235.57,190.42) rectangle (239.56,194.40);

\path[draw=drawColor,line width= 0.4pt,line join=round,line cap=round] (253.07,208.46) rectangle (257.06,212.45);

\path[draw=drawColor,line width= 0.4pt,line join=round,line cap=round] (270.58,230.79) rectangle (274.57,234.78);

\path[draw=drawColor,line width= 0.4pt,line join=round,line cap=round] (288.09,252.02) rectangle (292.07,256.01);

\path[draw=drawColor,line width= 0.4pt,line join=round,line cap=round] (305.59,276.05) rectangle (309.58,280.03);

\path[draw=drawColor,line width= 0.4pt,line join=round,line cap=round] (323.10,303.31) rectangle (327.09,307.30);

\path[draw=drawColor,line width= 0.4pt,line join=round,line cap=round] (340.60,327.46) rectangle (344.59,331.44);

\path[draw=drawColor,line width= 0.4pt,line join=round,line cap=round] (358.11,356.19) rectangle (362.10,360.17);

\path[draw=drawColor,line width= 0.4pt,line join=round,line cap=round] (375.62,388.65) rectangle (379.60,392.64);

\path[draw=drawColor,line width= 0.4pt,line join=round,line cap=round] (393.12,424.26) rectangle (397.11,428.24);

\path[draw=drawColor,line width= 0.4pt,line join=round,line cap=round] ( 55.50,442.04) rectangle (204.10,382.04);

\path[draw=drawColor,line width= 0.4pt,line join=round,line cap=round] ( 58.20,430.04) -- ( 76.20,430.04);

\path[draw=drawColor,line width= 0.4pt,dash pattern=on 4pt off 4pt ,line join=round,line cap=round] ( 58.20,418.04) -- ( 76.20,418.04);

\path[draw=drawColor,line width= 0.4pt,dash pattern=on 1pt off 3pt ,line join=round,line cap=round] ( 58.20,406.04) -- ( 76.20,406.04);

\path[draw=drawColor,line width= 0.4pt,dash pattern=on 1pt off 3pt on 4pt off 3pt ,line join=round,line cap=round] ( 58.20,394.04) -- ( 76.20,394.04);

\path[draw=drawColor,line width= 0.4pt,line join=round,line cap=round] ( 67.20,430.04) circle (  2.25);

\path[draw=drawColor,line width= 0.4pt,line join=round,line cap=round] ( 67.20,418.04) circle (  2.25);

\path[draw=drawColor,line width= 0.4pt,line join=round,line cap=round] ( 65.21,404.05) rectangle ( 69.20,408.04);

\path[draw=drawColor,line width= 0.4pt,line join=round,line cap=round] ( 65.21,392.05) rectangle ( 69.20,396.04);

\node[text=drawColor,anchor=base west,inner sep=0pt, outer sep=0pt, scale=  1.00] at ( 85.20,426.60) {Test Machine 1, $\mathbb{F}_{97}\qquad$};

\node[text=drawColor,anchor=base west,inner sep=0pt, outer sep=0pt, scale=  1.00] at ( 85.20,414.60) {Test Machine 1, $\mathbb{F}_{2^{256}}\qquad$};

\node[text=drawColor,anchor=base west,inner sep=0pt, outer sep=0pt, scale=  1.00] at ( 85.20,402.60) {Test Machine 2, $\mathbb{F}_{97}\qquad$};

\node[text=drawColor,anchor=base west,inner sep=0pt, outer sep=0pt, scale=  1.00] at ( 85.20,390.60) {Test Machine 2, $\mathbb{F}_{2^{256}}\qquad$};
\end{scope}
\end{tikzpicture}

  \caption{Evaluation Time of Polynomial (Larger Degrees) on Both Test Machines}
  \label{fig:poly-deg-t}
\end{figure}

\begin{figure}[ht]
  \centering
  % Created by tikzDevice version 0.6.2-92-0ad2792 on 2012-12-05 15:12:40
% !TEX encoding = UTF-8 Unicode
\begin{tikzpicture}[x=1pt,y=1pt]
\definecolor[named]{fillColor}{rgb}{1.00,1.00,1.00}
\path[use as bounding box,fill=fillColor,fill opacity=0.00] (0,0) rectangle (433.62,216.81);
\begin{scope}
\path[clip] ( 49.20, 61.20) rectangle (408.42,167.61);
\definecolor[named]{drawColor}{rgb}{0.00,0.00,0.00}

\path[draw=drawColor,line width= 0.4pt,line join=round,line cap=round] ( 62.50, 65.14) --
	( 76.36, 67.22) --
	( 90.22, 70.68) --
	(104.08, 74.32) --
	(117.94, 77.78) --
	(131.80, 80.90) --
	(145.66, 84.88) --
	(159.52, 89.04) --
	(173.37, 93.37) --
	(187.23, 98.04) --
	(201.09,101.85) --
	(214.95,105.83) --
	(228.81,108.60) --
	(242.67,113.63) --
	(256.53,117.61) --
	(270.39,122.80) --
	(284.25,127.48) --
	(298.10,135.10) --
	(311.96,136.66) --
	(325.82,141.16) --
	(339.68,144.79) --
	(353.54,150.68) --
	(367.40,153.11) --
	(381.26,157.26) --
	(395.12,163.67);

\path[draw=drawColor,line width= 0.4pt,line join=round,line cap=round] ( 62.50, 65.14) circle (  2.25);

\path[draw=drawColor,line width= 0.4pt,line join=round,line cap=round] ( 76.36, 67.22) circle (  2.25);

\path[draw=drawColor,line width= 0.4pt,line join=round,line cap=round] ( 90.22, 70.68) circle (  2.25);

\path[draw=drawColor,line width= 0.4pt,line join=round,line cap=round] (104.08, 74.32) circle (  2.25);

\path[draw=drawColor,line width= 0.4pt,line join=round,line cap=round] (117.94, 77.78) circle (  2.25);

\path[draw=drawColor,line width= 0.4pt,line join=round,line cap=round] (131.80, 80.90) circle (  2.25);

\path[draw=drawColor,line width= 0.4pt,line join=round,line cap=round] (145.66, 84.88) circle (  2.25);

\path[draw=drawColor,line width= 0.4pt,line join=round,line cap=round] (159.52, 89.04) circle (  2.25);

\path[draw=drawColor,line width= 0.4pt,line join=round,line cap=round] (173.37, 93.37) circle (  2.25);

\path[draw=drawColor,line width= 0.4pt,line join=round,line cap=round] (187.23, 98.04) circle (  2.25);

\path[draw=drawColor,line width= 0.4pt,line join=round,line cap=round] (201.09,101.85) circle (  2.25);

\path[draw=drawColor,line width= 0.4pt,line join=round,line cap=round] (214.95,105.83) circle (  2.25);

\path[draw=drawColor,line width= 0.4pt,line join=round,line cap=round] (228.81,108.60) circle (  2.25);

\path[draw=drawColor,line width= 0.4pt,line join=round,line cap=round] (242.67,113.63) circle (  2.25);

\path[draw=drawColor,line width= 0.4pt,line join=round,line cap=round] (256.53,117.61) circle (  2.25);

\path[draw=drawColor,line width= 0.4pt,line join=round,line cap=round] (270.39,122.80) circle (  2.25);

\path[draw=drawColor,line width= 0.4pt,line join=round,line cap=round] (284.25,127.48) circle (  2.25);

\path[draw=drawColor,line width= 0.4pt,line join=round,line cap=round] (298.10,135.10) circle (  2.25);

\path[draw=drawColor,line width= 0.4pt,line join=round,line cap=round] (311.96,136.66) circle (  2.25);

\path[draw=drawColor,line width= 0.4pt,line join=round,line cap=round] (325.82,141.16) circle (  2.25);

\path[draw=drawColor,line width= 0.4pt,line join=round,line cap=round] (339.68,144.79) circle (  2.25);

\path[draw=drawColor,line width= 0.4pt,line join=round,line cap=round] (353.54,150.68) circle (  2.25);

\path[draw=drawColor,line width= 0.4pt,line join=round,line cap=round] (367.40,153.11) circle (  2.25);

\path[draw=drawColor,line width= 0.4pt,line join=round,line cap=round] (381.26,157.26) circle (  2.25);

\path[draw=drawColor,line width= 0.4pt,line join=round,line cap=round] (395.12,163.67) circle (  2.25);
\end{scope}
\begin{scope}
\path[clip] (  0.00,  0.00) rectangle (433.62,216.81);
\definecolor[named]{drawColor}{rgb}{0.00,0.00,0.00}

\path[draw=drawColor,line width= 0.4pt,line join=round,line cap=round] ( 62.50, 61.20) -- (339.68, 61.20);

\path[draw=drawColor,line width= 0.4pt,line join=round,line cap=round] ( 62.50, 61.20) -- ( 62.50, 55.20);

\path[draw=drawColor,line width= 0.4pt,line join=round,line cap=round] (131.80, 61.20) -- (131.80, 55.20);

\path[draw=drawColor,line width= 0.4pt,line join=round,line cap=round] (201.09, 61.20) -- (201.09, 55.20);

\path[draw=drawColor,line width= 0.4pt,line join=round,line cap=round] (270.39, 61.20) -- (270.39, 55.20);

\path[draw=drawColor,line width= 0.4pt,line join=round,line cap=round] (339.68, 61.20) -- (339.68, 55.20);

\node[text=drawColor,anchor=base,inner sep=0pt, outer sep=0pt, scale=  1.00] at ( 62.50, 39.60) {0};

\node[text=drawColor,anchor=base,inner sep=0pt, outer sep=0pt, scale=  1.00] at (131.80, 39.60) {50};

\node[text=drawColor,anchor=base,inner sep=0pt, outer sep=0pt, scale=  1.00] at (201.09, 39.60) {100};

\node[text=drawColor,anchor=base,inner sep=0pt, outer sep=0pt, scale=  1.00] at (270.39, 39.60) {150};

\node[text=drawColor,anchor=base,inner sep=0pt, outer sep=0pt, scale=  1.00] at (339.68, 39.60) {200};

\path[draw=drawColor,line width= 0.4pt,line join=round,line cap=round] ( 49.20, 70.51) -- ( 49.20,157.09);

\path[draw=drawColor,line width= 0.4pt,line join=round,line cap=round] ( 49.20, 70.51) -- ( 43.20, 70.51);

\path[draw=drawColor,line width= 0.4pt,line join=round,line cap=round] ( 49.20, 87.83) -- ( 43.20, 87.83);

\path[draw=drawColor,line width= 0.4pt,line join=round,line cap=round] ( 49.20,105.14) -- ( 43.20,105.14);

\path[draw=drawColor,line width= 0.4pt,line join=round,line cap=round] ( 49.20,122.46) -- ( 43.20,122.46);

\path[draw=drawColor,line width= 0.4pt,line join=round,line cap=round] ( 49.20,139.77) -- ( 43.20,139.77);

\path[draw=drawColor,line width= 0.4pt,line join=round,line cap=round] ( 49.20,157.09) -- ( 43.20,157.09);

\node[text=drawColor,rotate= 90.00,anchor=base,inner sep=0pt, outer sep=0pt, scale=  1.00] at ( 34.80, 70.51) {0.1};

\node[text=drawColor,rotate= 90.00,anchor=base,inner sep=0pt, outer sep=0pt, scale=  1.00] at ( 34.80,105.14) {0.3};

\node[text=drawColor,rotate= 90.00,anchor=base,inner sep=0pt, outer sep=0pt, scale=  1.00] at ( 34.80,139.77) {0.5};

\path[draw=drawColor,line width= 0.4pt,line join=round,line cap=round] ( 49.20, 61.20) --
	(408.42, 61.20) --
	(408.42,167.61) --
	( 49.20,167.61) --
	( 49.20, 61.20);
\end{scope}
\begin{scope}
\path[clip] (  0.00,  0.00) rectangle (433.62,216.81);
\definecolor[named]{drawColor}{rgb}{0.00,0.00,0.00}

\node[text=drawColor,anchor=base,inner sep=0pt, outer sep=0pt, scale=  1.00] at (228.81, 15.60) {Polynomial Degree};

\node[text=drawColor,rotate= 90.00,anchor=base,inner sep=0pt, outer sep=0pt, scale=  1.00] at ( 10.80,114.41) {Running Time [s]};
\end{scope}
\end{tikzpicture}

  \caption{Evaluation Time of Polynomial (Smaller Degrees) on Test Machine 1}
  \label{fig:poly-deg-t-small}
\end{figure}

\begin{table}[ht]
  \centering
  \begin{tabular}{|c|c|}
    Polynomial Degree & Average Running Time [s] \\
    0 & 0.0694 \\
    100 & 0.2772 \\
    200 & 0.5434 \\
    300 & 0.8466 \\
    400 & 1.2524 \\
    500 & 1.6181 \\
    1000 &  3.9952 \\
    1500 &  6.9746 \\
    2000 & 10.8058 \\
    2500 & 12.8614 \\
    3000 & 17.3338 \\
    3500 & 21.2246 \\
    4000 & 26.8632 \\
    4500 & 29.8446 \\
    5000 & 34.9154 \\
    5500 & 41.0512 \\
    6000 & 48.2210 \\
  \end{tabular}
  \caption{Evaluation Time of Polynomial (Larger Degrees) on Test Machine 1}
  \label{tab:poly-deg-t}
\end{table}

\begin{table}[ht]
  \centering
  \begin{tabular}{|c|c|}
    Polynomial Degree & Average Running Time [ms] \\
    0  & 68.4 \\
    10 &  80.2 \\
    20 & 101.2 \\
    30 & 121.6 \\
    40 & 142.6 \\
    50 & 160.8 \\
    60 & 183.0 \\
    70 & 208.0 \\
    80 & 232.2 \\
    90 & 258.4 \\
    100 & 280.4 \\
    110 & 307.2 \\
    120 & 321.2 \\
    130 & 347.2 \\
    140 & 372.8 \\
    150 & 402.6 \\
    160 & 429.4 \\
    170 & 470.8 \\
    180 & 484.4 \\
    190 & 508.0 \\
    200 & 527.8 \\
    210 & 563.8 \\
    220 & 601.6 \\
    230 & 606.8 \\
    240 & 640.0 \\
  \end{tabular}
  \caption{Evaluation Time of Polynomial (Smaller Degrees) on Test Machine 1}
  \label{tab:poly-deg-t-small}
\end{table}

\begin{JWtodoBox}

\begin{itemize}

\item Wie schnell ist die derzeitige Implementierung?

\item Wie groß kann das Polynom werden?

\end{itemize}

\end{JWtodoBox}

\JWltwo{Contribution}

\JWltwo{What's left?}

Implementing David \& Goliath protocol. The implementation assumes it's already
available.

% vim: set spell spelllang=en_us fileencoding=utf8 :
