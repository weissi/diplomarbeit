\JWlone{Conclusion}
\label{sec:conclusion}

This thesis contributes a protocol implementing \emph{Oblivious Polynomial
Evaluation} in linear time. The protocol is implemented and proven to be
UC--secure \cite{canetti05} against passively and actively corrupted parties.
Hence, the protocol can be used ideally suited as a cryptographic primitive to
build future work on.

For restricted security requirements, evaluation of arbitrary arithmetic
circuits (including Square \& Multiply \cite{knuth81}) is possible using the
techniques presented. More specifically, restricted security requirements means
security against an actively corrupted evaluating party (\JWpTwo{}) but security
against a passively corrupted function definition party (\JWpOne{}) only.


%
% OPEN QUESTIONS AND OUTLOOK
%
\JWltwo{Open Questions and Outlook}
\label{sec:outlook}

The most important open question is to find an improvement for the current
technique to support arbitrary arithmetic circuits and not just polynomials
while maintaining the computing time complexity and the UC--security against all
actively corrupted parties.

The implementation would benefit from a better implementation of a library for
calculations in large finite fields such as $\mathbb{F}_{2^{k}}$ for $k > 128$.
Additionally, the David \& Goliath protocol \cite{davidgoliath} should be
implemented as a component the implementation of this thesis could use. The
current implementation assumes a tamper--proof hardware token that implements
the OAFE functionality (the program \JWBtoken{} that comes with this thesis
implements the functionality naively but insecure).

% vim: set spell spelllang=en_us fileencoding=utf8 formatoptions=tcroql : %
