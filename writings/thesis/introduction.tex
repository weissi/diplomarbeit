\JWlone{Introduction}
\label{sec:introduction}

\JWtodo{Erklärung was OAFEs sind}

\JWtodo{Effizienz, also lineare Zeit irgendwie reinbringen}

\JWtodo{Eigentlich neu schreiben, Kapitel ist totaler Schrott}

This thesis describes how two parties can securely evaluate an arithmetic
function such as $f(x_G,x_D) = ...$ over large finite fields. The main building
block are OAFEs as provided by the David \& Goliath protocol
\cite{davidgoliath}. However, the first step is to securely express the input
function $f(x_G, x_D)$ in terms of affine functions. These affine functions can
then be securely evaluated using OAFEs. At a first glance, the task of
expressing an arbitrary function $f(x_G, x_D)$ using affine functions may seem a
rather easy task. The challenge is to do it securely: The other party should not
be able to learn $G$ when it is given the affine functions expressing the partly
evaluated function $f(G, x_D)$ (value $G$ already set).  Additionally, the
process should work non--interactively: The first party fixes its input $x_G =
G$, partly evaluates $f(G, x_D)$, transforms it to affine functions and sends
them as OAFEs to the second party. The affine functions themselves are either
constant or hidden inside an OAFE calculation. After the second party provided
their input $x_D$ to the OAFEs, it gets (from the OAFEs) the evaluated results
of the affine functions. Using the described protocol, the second party is now
able to securely and fully evaluate $f(x_G, x_D)$. The whole computation is
expressed in terms of OAFE computations. Neither party will learn more of the
other party's input than the final result will tell it anyway.


\JWltwo{Related Work}
\label{sec:related-work}

\JWtodo{Hauptsächlich das von chapter \ref{sec:discontinued}}


\JWltwo{High--level Description}

\begin{JWtodoBox}

\begin{itemize}

\item Motivation

\item Praktischer Anwendungsfall

\item Was ist das Problem?

\item Was will ich erreichen?

\item High--level description of my thesis

\end{itemize}

\end{JWtodoBox}
