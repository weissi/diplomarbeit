\JWlone{Introduction}
\label{sec:introduction}

An \JWdef{Secure Multi--Party Computation}{MPC} or \JWdef{Secure Function
Evaluation}{SFE} is the joint calculation of an arbitrary function $f$ with
private inputs from a set of mutually distrusting parties. If the size of the
set of parties for an MPC is $2$, the computation is also called a two--party
computation. The area of research of secure multi--party computations was
founded by \emph{Yao's Millionaires' Problem} \cite{yao82}. The famous
Millionaires' Problem discusses two millionaires, who are interested in knowing
which of them is richer without revealing the actual values of their wealth. The
problem is evaluating the function $f(a, b) = a \geq b$, where the first
millionaire inputs the value of his wealth as $a$ and the second inputs $b$. The
result of the evaluation is either \textit{false} if the second millionaire is
richer or \textit{true} meaning the opposite. From the result and the own input,
neither party is able to calculate the wealth of the other millionaire. Another
example for real--world multi--party computations are democratic elections. The
eligible voters input their private vote and the result of the calculations is
the percentage breakdown.

\JWdefn{Oblivious Polynomial Evaluation}{OPE} \cite{naor99,naor06} is a special
case of SFE. In contrast to arbitrary functions that are known by the involved
parties, OPE only allows the first party to privately submit a polynomial $p$
and another party to evaluate the polynomial at one node $x$. The second party
obtains the result $p(x)$ without revealing $x$ to the first party or revealing
the polynomial $p$ to the second party. We find a secure and efficient OPE
protocol a useful and interesting primitive. \JWtodo{Beispiele warum OPE
wichtig.}

\JWdefn{Oblivious Affine Function Evaluation}{OAFE} \cite{davidgoliath} is a
primitive that allows---similar to OPE---to obliviously evaluate affine
functions. The David \& Goliath protocol \cite{davidgoliath} presents a secure
and efficient OAFE implementation based on only one stateful tamper--proof
hardware token. This thesis works out an efficient, secure, and implemented
protocol which realizes efficient OPE based on OAFE. The presented approaches
are also suitable for general SFE (including \emph{Square \& Multiply}
\cite{knuth81}), however the security is only proved for OPE. The security of
the protocol against passive and active adversaries is
in\-for\-ma\-tion--the\-o\-ret\-ically proved in the \JWdefn{Universal
Composability}{UC} framework \cite{canetti05}. \JWtodo{Hier alle Beiträge meiner
Arbeit reischreiben}


%
% RELATED WORK
%
\JWltwo{Related Work}
\label{sec:related-work}

The notion of \emph{universal composability} in the UC framework by Canetti
\cite{canetti05} places strict demands on the security of cryptographic
protocols. Proofs in the UC framework have to show that any environment is
unable to distinguish between an ideal functionality and the actually
implemented protocol, even when malicious adversaries misuse the protocol in
some unpredictable way. The benefit of the strict requirements of this notion is
that UC--secure protocols can be composed and ran concurrently while still
staying safe without the necessity of additional security proofs. This thesis
states universal composability as the notion of security.

Yao defined the problem of multi--party computations and his \JWdefn{Garbled
Circuit}{GC} approach \cite{yao86} describes encrypted evaluation of (boolean)
circuits. But despite the versatility of Yao's garbled circuits, the approach is
not suitable for arithmetic functions on large fields. Because the GC approach
requires embedded truth tables in every gate, an adoption to larger finite
fields would have the consequence of a quadratic blowup of every truth table in
every gate (cf.\ \cite{naor99privacy}). This thesis solves these problem
efficiently for arbitrary polynomials.

Besides the GC approach, this thesis is also related to \emph{Efficient
Multi-Party Computation Over Rings} by Cramer et al.\cite{cramer03}. The
difference is that Cramer's approach is based on the simulation of formulas by
bounded--width programs by Cleve \cite{cleve91} which does not support
\emph{Square \& Multiply} \cite{knuth81} and the complexity is slightly
polynomial in the number of arithmetic operations performed. The writings of
Naor et al.\cite{naor99,naor06} also describe OPE but are in polynomial
complexity and based on \JWdefn{Oblivious Transfer}{OT} \cite{rabin81}.

\emph{How to Garble Arithmetic Circuits} by Applebaum et al.\cite{gac2012}
describes the garbled evaluation of arithmetic circuits.  Many ideas used in
this thesis are inspired by Applebaum et al. The difference between the
Applebaum's paper and this thesis, is that the former relies on computational
assumptions. In contrast, this thesis does not rely on computational assumptions
and proves the methodology to be information--theoretically secure. This thesis
switches some of Applebaum's ideas to \JWdefn{Oblivious Affine Function
Evaluation}{OAFE} \cite{davidgoliath}.

%
% ACKNOWLEDGMENTS
%
\JWltwo{Acknowledgments}

I wish to thank my advisor Daniel Kraschewski who supported me throughout the
research and writing of this thesis. He donated a large amount of his limited
time for discussions and explorations crucial for my thesis and allowed this
thesis to be my own work but steered me in the right direction when needed.  I
am also thankful for his patience introducing me to this exciting topic.


%
% Outline
%
\JWltwo{Outline}

This thesis starts by introducing the reader to the general methodology, which
is discussed in depth thereafter (Chapter \ref{sec:methods}). Next, the security
of the methodology is analyzed and proved in the UC framework (Chapters
\ref{sec:protocol} and \ref{sec:security}). Since this thesis also implements
the proposed protocol, the implementation is covered (Chapter
\ref{sec:implementation}). An evaluation of both, the methodology and the
implementation (Chapter \ref{sec:evaluation}) rounds out this thesis.  And since
a large amount of the time taken for this thesis was affected by researching, a
quick tour of the alternative approaches is given to the reader (Chapter
\ref{sec:discontinued}).

% vim: set spell spelllang=en_us fileencoding=utf8 formatoptions=tcroql : %
