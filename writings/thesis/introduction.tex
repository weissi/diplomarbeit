\JWlone{Introduction}
\label{sec:introduction}

A \JWdef{Secure Multi--party Computation}{MPC} or \JWdef{Secure Function
Evaluation}{SFE} is the joint calculation of an arbitrary function $f$ with
private inputs from a set of mutually distrusting parties. The function has to
be evaluated correctly and the inputs have to be kept privately.  When the size
of the set of parties for a MPC is $2$, the computation is also called a
\emph{secure two--party computation}. The area of research of secure
multi--party computations was founded by \emph{Yao's Millionaires' Problem}
\cite{yao82}. The famous Millionaires' Problem discusses two millionaires, who
are interested in knowing which of them is richer without revealing the actual
values of their wealth. The problem is evaluating the function $f(a, b) = a \geq
b$, where the first millionaire inputs the value of his wealth as $a$ and the
second inputs $b$. The result of the evaluation is either $false$ so the second
millionaire is richer or $true$ meaning the opposite. From the result and the
own input, neither party is able to calculate the wealth of the other
millionaire. Another example for real--world multi--party computations are
democratic elections. The eligible voters input their private vote and the
result of the calculations is the percentage breakdown.

\JWdefn{Oblivious Polynomial Evaluation}{OPE} \cite{naor99,naor06} is a special
case of SFE. In contrast to arbitrary functions that are known by the involved
parties, OPE only allows the first party to privately submit a polynomial $p$
and another party to evaluate the polynomial at one node $x$. The second party
obtains the result $p(x)$ without revealing $x$ to the first party or revealing
the polynomial $p$ to the second party. We find a secure and efficient OPE
protocol a useful and interesting primitive.

\JWdefn{Oblivious Affine Function Evaluation}{OAFE} \cite{davidgoliath} is a
primitive that allows---similar to OPE---to obliviously evaluate affine
functions. This thesis works out an efficient, secure and implemented protocol
for OPE based on OAFE. This thesis also works out approaches for general SFE.
The security of the protocol is information--theoretically proved in the
\JWdefn{Universal Composability}{UC} framework \cite{canetti05}.


%
% RELATED WORK
%
\JWltwo{Related Work}
\label{sec:related-work}

The notion of security used in this thesis is \emph{universal composability} as
defined by Canetti \cite{canetti05}. The benefit of this notion is that
UC--secure protocols can be composed and ran concurrently while still staying
safe without the necessity of additional security proofs. Besides the notion of
security, Yao's \JWdefn{Garbled Circuit}{GC} approach \cite{yao86} is also
related to this thesis because Yao garbles boolean circuits to garbled circuits.
Despite the wide range of adoptions of Yao's garbled circuits, the approach is
not suitable for arithmetic functions on large fields. Because the GC approach
requires embedded truth tables in every gate, an adoption to larger finite
fields would have the cost of a quadratic blowup of every truth table in every
gate (cf.\ \cite{naor99privacy}).

Besides the GC approch, this thesis is also related to \emph{Efficient
Multi-party Computation Over Rings} by Cramer et al.\cite{cramer03}. The
difference is that Cramer's approach is based on the simulation of formalas by
bounded--width programs by Cleve \cite{cleve91} which does not support square \&
multiply \cite{knuth81} and the complexity is slightly polynomial in the number
of arithmetic operations performed. The writings of Naor et
al.\cite{naor99,naor06} also describe OPE but are in polynomial complexity and
based on \JWdefn{Oblivious Transfer}{OT} \cite{rabin81}.

\emph{How to Garble Arithmetic Circuits} by Applebaum et al.\cite{gac2012} is
related closely to this. Many ideas used in this thesis are inspired by
Applebaum et al. The difference between the paper and this thesis, is that
Applebaum et al.\ rely on computational assumptions. In contrast, this thesis
does not rely on computational assumptions but rebases some of the ideas on
\JWdefn{Oblivious Affine Function Evaluation}{OAFE} \cite{davidgoliath}. OAFE
allows oblivious evaluation of affine functions. This primitive is used to
encode arbitrary functions.

% vim: set spell spelllang=en_us fileencoding=utf8 formatoptions=tcroql : %
