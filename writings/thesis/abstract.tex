\noindent{}\textsf{\textbf{\huge Zusammenfassung}}

\bigskip{}

\begin{otherlanguage}{ngerman}

\noindent{}In der heutigen Zeit ist starke Kryptographie nicht mehr wegzudenken.
Vor allem, aber nicht ausschließlich, bei wichtigen Transaktionen über das
Internet oder andere Kommunikationsnetze ist es von herausragender Bedeutung,
dass Informationen zuverlässig, integer und vertraulich das vorgesehene Ziel
erreichen. Die Zuverlässigkeit wird in der Telematik untersucht. Die Integrität
und die Vertraulichkeit werden jedoch mit kryptographischen Methoden
sichergestellt. Da die Gesamtheit ein sehr komplexes Problem ist, geht man
zunehmend dazu über komplexe Verfahren auf kryptographischen Primitiven und
Protokollen zu basieren, die bewiesen sicher und komponierbar sind. Wie in
anderen Wissenschaften können die einzelnen Komponenten dann zum einem
funktionierenden Gesamtsystem zusammengesteckt werden.

Die vorliegende Arbeit beschäftigt sich mit kryptographischen Primitiven zur
sicheren Mehrparteienberechnung. Sichere Mehrparteienberechnungen sind
Berechnungen, die von mehreren Parteien durchgeführt werden, wobei keine Partei
mehr Wissen erlangt als sie aus ihrer Eingabe und der Ausgabe der Berechnung
ohnehin hätte berechnen können. Diese Arbeit beschäftigt sich hauptsächlich mit
Primitive \JWdefn{Oblivious Polynomial Evaluation}{OPE} die eine Unterklasse
der sicheren Mehrparteienberechnung darstellt. Mit OPE können zwei Parteien
gemeinsam ein Polynom auswerten wobei die erste Partei das Polynom festlegt und
keine Informationen erhält. Die zweite Partei soll das Polynom an einer
Stützstelle auswerten können und das Ergebnis, jedoch nicht das Polynom an sich,
erhalten. Weitere Informationen können weder die erste, noch die zweite Partei
errechnen. Außer der selbstverständlichen Anwendung von OPE zum Evaluieren von
Polynomen gibt es für die kryptographische Primitive OPE noch zahlreiche
Anwendungen, zum Beispiel das Generieren der \emph{Shares} bei \emph{Shamir's
Secret Sharing} \cite{shamir79}. Die in der Arbeit vorgestellten Methoden lassen
sich auch für die größere Klasse \JWdefn{Secure Function Evaluation}{SFE}, mit
der allgemeine Funktionen ausgewertet werden können anwenden, der Fokus liegt
jedoch auf OPE. Der Sicherheitsbeweis behandelt ebenfalls OPE, spätere Arbeiten
können aber die hier vorgestellten Verfahren auf SFE erweitern.

Da OPE eine komplexe Primitive ist, basiert diese Arbeit auf der
(schwächeren) Primitive \JWdefn{Oblivious Affine Function Evaluation}{OAFE} die
es ermöglicht affine Funktionen ebenso auszuwerten wie OPE es für Polynome
erlaubt.

Um dieses ambitionierte Ziel zu erreichen untersucht diese Arbeit verschiedene
Ansätze und kommt zu einem bemerkenswerten Ergebnis: Mit den in dieser Arbeit
vorgestellten Methoden und OAFE ist es möglich beliebige Polynome über großen
endlichen Körpern in Linearzeit auszuwerten. Das heißt, die Auswertung ist nur
um einen konstanten Faktor langsamer als die übliche Auswertung des Polynoms.
Die Sicherheit des Verfahrens wird unter dem strengen Sicherheitsbegriff der
\JWdefn{Universal Composability}{UC} \cite{canetti05} bewiesen. Neben der
theoretischen Ausarbeitung liegt dieser Thesis auch eine vollständige
Implementierung in der funktionalen Programmiersprache \JWThaskell{} bei, die
die angesprochenen Eigenschaften zeigt.

Die Arbeit wird letztendlich durch eine Evaluierung sowohl der theoretischen als
auch des praktischen Teils abgerundet.

\end{otherlanguage}


\cleardoublepage

\noindent{}\textsf{\textbf{\huge Abstract}}

\bigskip{}

\noindent{}Today, strong cryptography plays a very important role. Cryptography
is of crucial importance mainly, but not limited, to important transactions via
the Internet or other communication networks because the information should
reach its destination reliably, confidentially and with integrity. The
telematics research solves the reliability problem, but cryptography is used to
ensure confidentiality and integrity. Since the overall problem is very complex,
an increasing transition to base complex techniques on cryptographic primitives
and protocols is discoverable. The benefit of this component based architecture
is the possibility to prove the security of the components individually. Of
course, the proofs must concern the composability of the components that enables
to build complex and secure systems from smaller building blocks.

This thesis concerns itself with cryptographic primitives for \JWdefn{Secure
Multi--Party Computations}{MPC}. MPC are joint computations of a set parties
which reveal the result of the computation to any party and nothing more. Every
party should only learn information that can be calculated from its own input
and the result. This thesis mainly deals with \JWdefn{Oblivious Polynomial
Evaluation}{OPE}, a subset of general MPC. OPE allows two parties to jointly
evaluate a polynomial where the first party chooses the polynomial and learns
nothing. The second party chooses the node and only learns the evaluation of the
polynomial at the chosen node. In addition to evaluation of polynomials which
is an obvious use--case of OPE, OPE hat many interesting applications, such as
the share generation for \emph{Shamir's Secret Sharing} \cite{shamir79}. The
methodology of this thesis also applies to the larger class of \JWdefn{Secure
Function Evaluation}{SFE} which enables to securely evaluate arbitrary
functions. Though, the focus as well as the target of the security proof is OPE.
Further works might extend the methods to be entirely suitable for SFE.

Because OPE is a complex primitive, this thesis bases its implementation of the
protocol realizing OPE on the weaker cryptographic primitive \JWdefn{Oblivious
Affine Function Evaluation}{OAFE}. OAFE allows the evaluation of affine
functions in just the same way OPE enables to evaluate polynomials.

In order to reach this ambitious aim, this thesis examines various approaches
and comes up with the astonishing result of OPE in linear time. That is,
evaluation of polynomials as fast as usual evaluation except for a constant
factor. The security of the methodology is proved in the \JWdefn{Universal
Composability}{UC} framework \cite{canetti05} which places very strict demands
on the security of cryptographic protocols.

Alongside the theoretical debate, this thesis also features an efficient, secure
and working implementation in the functional programming language \JWThaskell{}
which manifests the properties of the protocol mentioned above.

This thesis is ultimately carried out by an evaluation of both, the theoretical
methodology and the implementation.

\cleardoublepage

% vim: set spell spelllang=en_us fileencoding=utf8 formatoptions=tcroql : %
