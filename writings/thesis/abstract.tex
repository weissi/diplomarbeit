\noindent{}\textsf{\textbf{\huge Abstract}}

\bigskip{}

\noindent{}Today, strong cryptography plays a very important role. Cryptography
is of crucial importance mainly, but not limited, to important transactions via
the Internet and other communication networks because the information should
reach its destination reliably, confidentially and with integrity. The
telematics research solves the reliability problem, but cryptography is used to
ensure confidentiality and integrity. Since the overall problem is very complex,
an increasing effort to use cryptographic primitives and protocols is
discoverable. The benefit of this component based architecture is the
possibility to prove the security of the components individually. Of course, the
proofs must take into account the composability of the components which enables
to build complex and secure systems from smaller building blocks.

This thesis concerns itself with cryptographic primitives for \JWdefn{Secure
Multi--Party Computations}{MPC}. MPC are joint computations of a set of parties
which reveal the result of the computation to any party and nothing else. Every
party should only learn information that can be calculated from its own input
and the result. This thesis mainly deals with \JWdefn{Oblivious Polynomial
Evaluation}{OPE}, a subset of general MPC. OPE allows two parties to jointly
evaluate a polynomial where the first party chooses the polynomial and learns
nothing. The second party chooses the node and only learns the evaluated result
of the polynomial at the chosen node. In addition to the evaluation of
polynomials which is an obvious use case of OPE, OPE has many interesting
applications, such as the share generation for \emph{Shamir's Secret Sharing}.
The methodology of this thesis also applies to the larger class of
\JWdefn{Secure Function Evaluation}{SFE} which enables to securely evaluate
arbitrary functions.

This thesis examines various approaches, leading to the novel result of a
cryptographic protocol realizing OPE in linear time. The security of the
methodology is proved in the \JWdefn{Universal Composability}{UC} framework
which places very strict demands on the security of cryptographic protocols.

Alongside the theoretical debate, this thesis also features an efficient, secure
and working implementation which manifests the properties of the protocol
mentioned above.

\cleardoublepage

\noindent{}\textsf{\textbf{\huge Zusammenfassung}}

\bigskip{}
\begin{otherlanguage}{ngerman}

\noindent{}In der heutigen Zeit ist starke Kryptographie nicht mehr wegzudenken.
Vor allem, aber nicht ausschließlich, bei wichtigen Transaktionen über das
Internet oder andere Kommunikationsnetze ist es von herausragender Bedeutung,
dass Informationen zuverlässig, integer und vertraulich das vorgesehene Ziel
erreichen. Die Zuverlässigkeit wird in der Telematik untersucht. Die Integrität
und die Vertraulichkeit werden jedoch mit kryptographischen Methoden
sichergestellt. Da die Gesamtheit ein sehr komplexes Problem ist, geht man
zunehmend dazu über komplexe Verfahren auf kryptographischen Primitiven und
Protokollen aufzubauen, die bewiesen sicher und komponierbar sind. Wie in
anderen Wissenschaften können die einzelnen Komponenten dann zu einem
funktionierenden Gesamtsystem zusammengesteckt werden.

Die vorliegende Arbeit beschäftigt sich mit kryptographischen Primitiven zur
sicheren Mehrparteienberechnung. Sichere Mehrparteienberechnungen sind
Berechnungen, die von mehreren Parteien durchgeführt werden, wobei keine Partei
mehr Wissen erlangt als sie aus ihrer Eingabe und der Ausgabe der Berechnung
ohnehin hätte berechnen können. Diese Arbeit beschäftigt sich hauptsächlich mit
der Primitive \JWdefn{Oblivious Polynomial Evaluation}{OPE}, die eine
Unterklasse der sicheren Mehrparteienberechnung darstellt. Mit OPE können zwei
Parteien gemeinsam ein Polynom auswerten, wobei die erste Partei das Polynom
festlegt und keine Informationen erhält. Die zweite Partei soll das Polynom an
einer Stützstelle auswerten können und das Ergebnis, jedoch nicht das Polynom an
sich, erhalten. Darüber hinausgehende Informationen können weder die erste, noch
die zweite Partei errechnen. Neben der selbstverständlichen Anwendung von OPE
zum Evaluieren von Polynomen gibt es für die kryptographische Primitive OPE noch
zahlreiche weitere Anwendungen, zum Beispiel die Generierung der \emph{Shares}
bei \emph{Shamir's Secret Sharing}. Die in der Arbeit vorgestellten Methoden
lassen sich auch auf die größere Klasse \JWdefn{Secure Function
Evaluation}{SFE}, mit der allgemeine Funktionen ausgewertet werden können,
anwenden, der Fokus liegt jedoch auf OPE.

Um dieses ambitionierte Ziel zu erreichen untersucht diese Arbeit verschiedene
Ansätze und kommt zu einem bemerkenswerten Ergebnis: Mit den in dieser Arbeit
vorgestellten Methoden ist es möglich beliebige Polynome über großen endlichen
Körpern in Linearzeit auszuwerten. Die Sicherheit des Verfahrens wird unter dem
strengen Sicherheitsbegriff der \JWdefn{Universal Composability}{UC} bewiesen.
Neben der theoretischen Ausarbeitung liegt dieser Thesis auch eine vollständige
Implementierung bei, die die angesprochenen Eigenschaften zeigt.

\end{otherlanguage}

\cleardoublepage

% vim: set spell spelllang=en_us fileencoding=utf8 formatoptions=tcroql : %
