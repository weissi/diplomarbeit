\JWlone{Implementation}
\label{sec:implementation}

In addition to the mathematical methods (chapter \ref{sec:methods}), the
protocol description (chapter \ref{sec:protocol}) and the security proof of the
protocol in the UC framework (chapter \ref{sec:security}), this thesis contains
an implementation of the protocol. The implementation is written in the lazy,
functional programming language \JWTLhaskell{}. More specifically \JWThaskell{}
as implemented by the major \JWThaskell{} compiler the \JWTXLghc{} (\JWTghc{})
version \JWTVghc{}.

The implementation was designed to match the descriptions in the above chapters
as closely as possible. The implementation implements Oblivious Polynomial
Evaluation as explained in chapter \ref{def:OPE}.

Even the Haskell data--types such as \JWcode{DRAC}, \JWcode{DRAE}, \JWcode{RAE},
\JWcode{LinearExpr} match the names of the constructs described in chapter
\ref{sec:methods}: DRACs (chapter \ref{def:DRAC}), (chapter \ref{def:DRAE}) and
linear expressions (named $\mathcal{C}$ in chapter \ref{def:DRAE}). For
practical reasons the program also features \JWcode{RAC} (Randomized Affine
Circuits) and \JWcode{RAE} (Randomized Affine Encodings) which are the same as
their \emph{dual} counterparts but split up by the components. So while running
the program two \JWcode{RAE}s are derived from one \JWcode{DRAE}.

As in chapter \ref{def:OPE}, the oblivious polynomial evaluation consists of
three parties: The tamper--proof hardware issuing party \JWpOne{} (executable
program called \JWBpOne{}), the receiving party \JWpTwo{} (\JWBpTwo{}) and the
tamper--proof hardware \JWtoken{} (\JWBtoken{}). The individual programs talk to
each other using TCP/IP networking. The \JWtoken{} implements the OAFE
functionality \JWfuncSym{seq-ot}{OAFE} as provided by the David \& Goliath
protocol \cite{davidgoliath}. The \JWtoken{} does not actually implement the
David \& Goliath protocol but only it's interface. A real implementation using a
tamper--proof hardware token is left open to potential future works.


%
% COMMUNICATION CHANNELS
%
\JWltwo{Communication Channels and Their TCP Ports}
\label{sec:communication-channels}


% GOLIATH TO TOKEN
\JWlthree{\JWpOne{} To \JWtoken{}}
\label{sec:comm:goliath2Token}

On TCP port \JWport{23120} \JWpOne{} initiates a connection to the \JWtoken{}.
This connection is used to transfer the OAFE configuration to the tamper--proof
hardware token.


% DAVID TO TOKEN
\JWlthree{\JWpTwo{} To \JWtoken{}}
\label{sec:comm:david2token}

On TCP port \JWport{23021} \JWpTwo{} initiates a connection to the \JWtoken{}.
This connection is used to evaluate the OAFEs. \JWpTwo{} sends a tells which
OAFE to evaluate and its value, the \JWtoken{} then replies with a vector
consisting of the evaluated linear expressions.


% GOLIATH TO DAVID
\JWlthree{\JWpOne{} To \JWpTwo{}}
\label{sec:comm:goliath2david}

On TCP port \JWport{23102} \JWpOne{} initiates a connection to \JWpTwo{}. This
connection is a stream of \JWpTwo{} DRAEs (in fact \JWcode{RAE}s) assigned to
variables, that \JWpTwo{} will evaluate one after the other in the order
\JWpOne{} has sent them.


% DAVID TO GOLIATH
\JWlthree{\JWpTwo{} To \JWpOne{}}
\label{sec:comm:david2goliath}

On TCP port \JWport{23201} \JWpTwo{} initiates a connection to \JWpOne{}. This
connection is not necessary for the protocol itself but is used to exchange
settings between \JWpOne{} and \JWpTwo{}. \JWpOne{} tells \JWpTwo{} the host
name and the TCP port of the \JWtoken{} and \JWpTwo{} tells \JWpOne{} the port
on which \JWpTwo{} accepts the DRAE stream.


%
% DIFFERENCES BETWEEN IMPLEMENTATION AND WRITING
%
\JWltwo{Differences Between the Implementation and this Writing}
\label{sec:implementation-differences}

\begin{itemize}

\item In writing there are no \JWcode{RAE}s

\item Communication Channel from \JWpTwo{} to \JWpOne{}
(\ref{sec:comm:david2goliath})

\end{itemize}

% vim: set spell spelllang=en_us fileencoding=utf8 :
