\JWlone{Implementation}
\label{sec:implementation}

In addition to the formal methodology (Chapter~\ref{sec:methods}) and the
security proof of the protocol in the UC framework (Chapter \ref{sec:security}),
this thesis contains an implementation of the protocol. The implementation is
written in the lazy, functional programming language \JWTLhaskell{}, more
specifically \JWThaskell{} as implemented by the major \JWThaskell{} compiler,
the \JWTXLghc{} (\JWTghc{}) version \JWTVghc{}. Whenever feasible, standard
\emph{Haskell 2010} \cite{haskell2010} features have been preferred. However,
the following extensions have been used: \JWhsExt{Bang\-Patterns},
\JWhsExt{Scoped\-Type\-Variables}, \JWhsExt{Rank2\-Types}, and
\JWhsExt{Flexible\-Contexts}.  The \JWTprotobuf{} (see
Chapter~\ref{sec:net-comm}) serialization and parsing code that is generated
automatically by \JWTLhprotoc{} additionally uses
\JWhsExt{Derive\-Data\-Typeable}, \JWhsExt{Flexible\-Instances}, and
\JWhsExt{Multi\-Param\-Type\-Classes}. The code should therefore be compilable
by any \JWThaskell{} compiler that supports these extensions and is able to
compile the external libraries.

The implementation is designed to match the descriptions in chapter
\ref{sec:methods} as closely as possible, it implements Oblivious Polynomial
Evaluation as explained in Chapter \ref{def:OPE}.

Even the Haskell data--types such as \JWcode{DRAC}, \JWcode{DRAE}, \JWcode{RAE},
\JWcode{LinearExpr} match the names of the constructs described in chapter
\ref{sec:methods}: DRACs (Chapter \ref{def:DRAC}), DRAEs (chapter
\ref{def:DRAE}) and linear expressions.

As in Chapter \ref{def:OPE}, the oblivious polynomial evaluation consists of
three parties: The OAFE issuing party \JWpOne{} (executable program called
\JWBpOne{}), the receiving party \JWpTwo{} (\JWBpTwo{}) and the OAFE
functionality (\JWBtoken{}). The individual programs talk to each other using
TCP/IP networking. Even though \JWBtoken{} implements the OAFE functionality
\JWfuncSymOAFE{}, it does not implement the David \& Goliath protocol
\cite{davidgoliath} but only its interface. A real implementation using a
tamper--proof hardware token is left open to potential future work.


%
% COMMUNICATION CHANNELS
%
\JWltwo{Communication Channels}
\label{sec:communication-channels}

The binaries \JWBpOne{}, \JWBpTwo{}, and \JWBtoken{} use TCP/IP networking to
communicate. In the following, the different TCP ports and the communication
which takes part over these channels are described.


% GOLIATH TO TOKEN
\JWlthree*{\JWpOne{} to \JWtoken{}}

On TCP port \JWport{23120} \JWpOne{} initiates a connection to the \JWtoken{}.
This connection serves to transfer the OAFE configuration. After having
successfully received the OAFE configuration, the OAFE functionality accepts
OAFE evaluation requests (usually from \JWpTwo{}) but no more OAFE
configurations. One \JWtoken{} process can be configured to a OAFE configuration
exactly once.


% DAVID TO TOKEN
\JWlthree*{\JWpTwo{} to \JWtoken{}}

On TCP port \JWport{23021} \JWpTwo{} initiates a connection to the \JWtoken{}.
This connection serves to evaluate the OAFEs. \JWpTwo{} tells which
OAFE to evaluate and the value. The \JWtoken{} then replies with a vector
consisting of the values of the evaluated linear expressions.


% DAVID TO GOLIATH
\JWlthree*{\JWpTwo{} to \JWpOne{}}

On TCP port \JWport{23201} \JWpTwo{} initiates a connection to \JWpOne{}. This
connection is not necessary for the protocol itself but serves to exchange
settings between \JWpOne{} and \JWpTwo{}. \JWpOne{} tells \JWpTwo{} the host
name and the TCP port of the \JWtoken{} and \JWpTwo{} tells \JWpOne{} the port
on which \JWpTwo{} accepts the DRAC.


% GOLIATH TO DAVID
\JWlthree*{\JWpOne{} to \JWpTwo{}}

On TCP port \JWport{23102} \JWpOne{} initiates a connection to \JWpTwo{}. This
connection streams the DRAC that \JWpTwo{} will evaluate.


%
% DIFFERENCES BETWEEN IMPLEMENTATION AND WRITING
%
\JWltwo{Differences Between the Implementation and this Writing}
\label{sec:implementation-differences}

The main difference between the current implementation and this writing is that
the OAFE functionality is not implemented using the David \& Goliath
\cite{davidgoliath} protocol. Instead, the \JWBtoken{} implements a naive
implementation of the ideal OAFE functionality. Further works could change the
implementation to a real implementation of the protocol. Another difference is
that \JWpTwo{} does not verify the polynomial degree it evaluates.


%
% IMPLEMENTATION DETAILS
%
\JWltwo{Implementation Details}
\label{sec:implementation-details}

This section describes implementation details that could be of interest or
need documentation.

% Calculations in Finite Fields
\JWlthree{Calculations in Finite Fields}
\label{sec:finite-field-calcs}

For calculations in finite fields, the \JWTcpp{} library \JWTLntl{}
(\emph{Number Theory Library}) has been interfaced to \JWThaskell{} to use in
the implementation of this thesis. The interface has been developed using
\JWThaskell{}'s \JWdef{Foreign Function Interface}{FFI} \cite{haskell2010} and
\JWdef{C $\longrightarrow$ Haskell}{C2Hs} \cite{c2hs}.  Before interfacing the
library to Haskell a very lightweight \JWTc{} wrapper has been developed because
the library is written in \JWTcpp{} which cannot directly be interfaced by the
FFI.

Although \JWTntl{} does its work fast and without problems, the library has some
implementation problems and is not optimally designed for exposing its
functionality to functional programming languages. First, the library is
neither thread--safe nor reentrant because it globally stores internal state
that can be modified using library functions. The small \JWTc{} wrapper library
therefore tries to provide a thread--safe interface to \JWTntl{} by limiting its
functionality.  Another problem is that \JWTntl{} is designed for mutable
objects, for example, arithmetic operations are implemented using destructive
assignment.  Since \JWThaskell{} is a pure functional language
\cite{haskell2010} and destructive assignment is referentially opaque, the
\JWTc{} wrapper changes that and offers only an interface that obeys referential
transparency. It does so by internally allocating a new element for the result
of arithmetic operations and other functions that would modify existing objects
otherwise.

The interface that obeys referential transparency makes the binding work well
inside \JWThaskell{} programs. But for large number of \JWTntl{} objects, there
is a performance problem: Creating and destructing a massive amount of objects
is slow. The problem is that it is not possible to allocate memory for \JWTntl{}
objects completely externally because \JWTntl{} allocates memory internally on
the heap in the constructor. In more detail: The objects the programmer gets
from the \JWTntl{} library are only one machine word wide which stores a pointer
to memory on the heap which stores the representation of the elements in the
finite field. In other words: Allocating memory for an array of field elements
is not in $O(1)$ as usual (a chunk of stack space or one \JWcode{malloc()} call)
but in $O(n)$ because \JWTntl{} allocates additional memory for every element on
the heap internally. Obviously, the same problem occurs when destructing an
array of \JWTntl{} objects: Usually that is in $O(1)$ as well (waiting until the
stack frame gets destroyed or \JWcode{free()} is called) but using \JWTntl{}
this is in $O(n)$ because \JWTcpp{}'s \JWcode{delete} has to be called for every
element. Not calling \JWcode{delete} will lead to a massive memory leak.
Therefore, the \JWThaskell{} binding cannot manage memory as usual but has to
call a special memory freeing function of the \JWTc{} wrapper that calls
\JWTcpp{}'s \JWcode{delete} for every object. This becomes a major performance
problem for evaluations of large polynomials (see chapter
\ref{sec:comp-complexity}). Nevertheless, this problem is easily fixable by
using a \JWThaskell{} library for finite field calculations or interfacing a
library that fits better.

To benchmark this thesis (see Chapter \ref{sec:evaluation}), a second library
(in $\mathbb{F}_{97}$, from the \JWTLhaskellForMaths{} package) has been used,
this library does not have the same problems as \JWTntl{}.


% Thread Safety
\JWlthree{Thread Safety}

The implementation itself is fully thread--safe, all communication is done using
\JWTghc{}'s implementation of \emph{Software Transactional Memory} (STM)
\cite{stm05} for Haskell. The library (\JWTntl{}) that is used for the
calculations in the finite fields is not thread--safe but the small \JWTc{}
wrapper (see Chapter \ref{sec:finite-field-calcs}) tries to fix \JWTntl{}'s
thread--safety problems. However, for guaranteed thread--safety \JWTntl{} should
be replaced by another library that is thread--safe. The \JWTntl{}
implementation also has other disadvantages that would make another library even
more useful (for details see Chapter \ref{sec:finite-field-calcs}).


% Testing Correctness
\JWlthree{Testing the Correctness of the Implementation}

The correctness of the implementation has not been proofed formally since that
is not feasible for complex programs. Instead of a real proof, the
implementation has been tested using the specification--driven automatic
\JWThaskell{} testing tool \JWTquickcheck{} \cite{quickcheck} and the classic
unit test approach of \JWTLhunit{}.


% Randomness
\JWlthree{Cryptographic Randomness}

The random numbers needed for the implementation of this thesis are generated
using \JWTLmonadcryptorandom{}. This package allows explicit failure of the
random number generation and exchangeable random number generators such as
\JWTLdrbg{}. \JWTdrbg{} implements a NIST standardized number--theoretically
secure random number generator.


% Communication Layer
\JWlthree{Network Communication Layer}
\label{sec:net-comm}

The communication between the different binaries is done using the
schrie\-bl\-esque \JWTXLprotobuf{}. This library features easy network
communication with complex data types independent of the programming language,
operating system, machine endianness, and other parameters. The \JWThaskell{}
\JWTprotobuf{} compiler \JWThprotoc{} that comes with the \JWThaskell{}
\JWTprotobuf{} library (\JWTprotobufLib{}) automatically generates the
\JWThaskell{} source code which serializes the data structures to
\JWTprotobuf{}, and code parses \JWTprotobuf{} to \JWThaskell{} data
structures. The automatically generated code can be found in the directory
\JWpath{gen-src/} in this thesis' source tree.


% Other Libraries
\JWlthree{Other Libraries}

This thesis uses many other libraries mainly from \JWTLhackage{}. The
complete list of packages can be extracted from the file
\JWpath{diplomarbeit.cabal} that comes with the source code of this thesis.


% Source Tree Organization
\JWlthree{Source Tree Organization}
\label{sec:src-org}

A listing of the most important directories in the source code tree of this
thesis can be found below.

\begin{itemize}

  \item \JWpath{programs/}, source code for the binaries (\JWBpOne{},
    \JWBpTwo{} and \JWBtoken{}).

  \item \JWpath{lib/}, source code for the library functionality.

  \item \JWpath{lib/Functionality/}, reusable code that forms the binaries'
    functionality.

  \item \JWpath{lib/Data/OAFE/}, OAFE functionality implementation.

  \item \JWpath{lib/Data/RAE/}, RAE encoding, decoding and evaluation.

  \item \JWpath{lib/Math/}, finite field implementation and other
    mathematical functions.

  \item \JWpath{test/}, \JWTquickcheck{} tests and the unit tests.

  \item \JWpath{bench/}, benchmarking programs.

  \item \JWpath{scripts/}, helper scripts, mainly for benchmarking and testing.

  \item \JWpath{protos/}, \JWTprotobuf{} description files.

  \item \JWpath{gen-src/}, automatically generated source code, derived from the
    \JWTprotobuf{} descriptions files.

  \item \JWpath{writings/}, this writing.

  \item \JWpath{evaluation/}, data for the figures of this writing.

  \item \JWpath{dist/}, after having compiled the source code, this folder
    contains the binaries.

\end{itemize}


% Build System
\JWlthree{Build System and Building}

To build the source code of this thesis, \JWTLcabal{} is used. \JWTcabal{}
is the standard building and packaging system used for \JWThaskell{} programs
and libraries. The build process can be started by executing the following
commands in a shell:

\JWcmd{cd diplomarbeit/}

\JWcmd{cabal install --only-dependencies\ \ \ \#install required packages}

\JWcmd{cabal configure}

\JWcmd{cabal build}


%
% USER INPUT/OUTPUT
%
\JWltwo{User Input/Output}
\label{sec:user-io}

Whenever field elements in large fields that are no prime fields (e.g.,
$\mathbb{F}_{2^{256}}$) are read from user input or are outputted to the user, a
special text format is used. Note that scalar field elements of non--prime
fields are elements from polynomial rings and therefore written as polynomials.
This implementation uses the following format (given as a regular expression) to
denote the polynomials (that represent the scalar field elements):

\JWcode{\textbackslash[([01]( [01])\{0,255\})?\textbackslash]}

\noindent{}Intuitively that is between $0$ and $256$ digits (each either
\JWcode{0} or \JWcode{1}) separated by spaces and surrounded by square brackets.
The meaning of such a string is $\sum_{p=0}^{P} d_p \cdot x^p$ where $d_p$ is
the digit at index $p$ in the string and $P$ is the maximal index (counted from
$0$, not counting the spaces and the brackets). An exception is the string
\JWcode{[]} which represents the polynomial \JWcode{0}.  Examples:

\begin{itemize}

  \item \JWcode{[]} and \JWcode{[0]} represent $0$

  \item \JWcode{[1]} represents $1 \cdot x^0 = 1 \cdot 1 = 1$

  \item \JWcode{[0 1]} represents $0 + 1 \cdot x^1 = x$

  \item \JWcode{[1 0]} represents $1 + 0 \cdot x^1 = 1$

  \item \JWcode{[1 1]} represents $1 + 1 \cdot x^1 = 1 + x$

  \item \JWcode{[1 0 1 0 1 0]} represents $1 + x^2 + x^4$

\end{itemize}

\noindent{}The default implementation uses $\mathbb{F}_{2^{256}}$ specified by
the irreducible polynomial $1 + x^2 + x^5 + x^{10} + x^{256}$.
$\mathbb{F}_{97}$ is a prime field and therefore the elements are regular
elements of the ring $\mathbb{Z}/2\mathbb{Z}$.  This implementation uses regular
numbers for any prime field. Exemplary for $\mathbb{F}_{97}$: $1 \entspricht 1
(mod 97) = 1_{\mathbb{F}_{97}}$, $98 \entspricht 98 (mod 97) =
1_{\mathbb{F}_{97}}$.


%
% USAGE OF THE PROGRAMS
%
\JWltwo{Usage of the Programs}
\label{sec:usage}

The three main programs are \JWBpOne{}, \JWBpTwo{} and \JWBtoken{}. \JWBtoken{}
does not accept any command line parameters. \JWpOne{} expects exactly one
command line parameter: The file with the polynomial to evaluate, one
coefficient per line. \JWpTwo{} expects exactly one parameter too: The field
element to evaluate the polynomial with. The following example would evaluate
the polynomial $f(x) = a \cdot 1 + b \cdot x + c \cdot x^2$ where $a = 1
\entspricht \text{\JWcode{[1]}}$, $b = 1 + p^2 \entspricht \text{\JWcode{[1 0
1]}}$, $c = (1 + p^2 + p^4) \entspricht \text{\JWcode{[1 0 1 0 1]}}$, and $x = 1
  + p^2 + p^3 + p^4 + p^6 \entspricht \text{\JWcode{[1 0 1 1 1 0 1]}}$:

\medskip{}

\JWcmd{(echo [1]; echo [1 0 1]; echo [1 0 1 0 1]) > /tmp/test.poly}

\JWcmd{Goliath /tmp/test.poly}

\JWcmd{Token}

\JWcmd{David '[1 0 1 1 1 0 1]'}

\medskip{}

\noindent{}The same procedure can be run by the provided
\JWpath{run-evaluation.sh} script that comes with this thesis:

\medskip{}

\JWcmd{run-evaluation.sh /tmp/test.poly '[1 0 1 1 1 0 1]'}

\medskip{}

\noindent{}Additionally, there is an implementation of the procedure in only one
binary, that \emph{does not} use TCP/IP networking. This optional version can be
used in the following way:

\medskip{}

\JWcmd{AllInOne /tmp/test.poly '[1 0 1 1 1 0 1]'}


%
% DOCUMENTATION
%
\JWltwo{Documentation}
\label{sec:implementation-doc}

The implementation is documented using \JWTLhaddock{} the standard tool for
automatically generating documentation from annotated \JWThaskell{} source code.
The generated HTML documentation for the implementation of this thesis is
available at:

\medskip{}

\noindent{}
\JWnamedlink{http://johannesweiss.eu/diplomarbeit/index.html}%
            {\texttt{http://johannesweiss.eu/diplomarbeit/index.html}}


%
% CODE AVAILABILITY
%
\JWltwo{Code Availability}
\label{sec:code-availability}

All of the code is open--sourced under the terms of the \JWTXLgpl{} and
available at \JWnamedlinkfn{https://github.com/weissi/diplomarbeit.git}{GitHub}.
The code can be obtained by executing the following command in a shell:

\JWcmd{git clone https://github.com/weissi/diplomarbeit.git}


% vim: set spell spelllang=en_us fileencoding=utf8 :
