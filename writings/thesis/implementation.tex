\JWlone{Implementation}
\label{sec:implementation}

In addition to the mathematical methods (chapter \ref{sec:methods}), the
protocol description (chapter \ref{sec:protocol}) and the security proof of the
protocol in the UC framework (chapter \ref{sec:security}), this thesis contains
an implementation of the protocol. The implementation is written in the lazy,
functional programming language \JWTLhaskell{}. More specifically \JWThaskell{}
as implemented by the major \JWThaskell{} compiler the \JWTXLghc{} (\JWTghc{})
version \JWTVghc{}.

The implementation was designed to match the descriptions in the above chapters
as closely as possible. The implementation implements Oblivious Polynomial
Evaluation as explained in chapter \ref{def:OPE}.

Even the Haskell data--types such as \JWcode{DRAC}, \JWcode{DRAE}, \JWcode{RAE},
\JWcode{LinearExpr} match the names of the constructs described in chapter
\ref{sec:methods}: DRACs (chapter \ref{def:DRAC}), DRAEs (chapter
\ref{def:DRAE}) and linear expressions (named $\mathcal{C}$ in chapter
\ref{def:DRAE}). For practical reasons the program also features \JWcode{RAC}
(Randomized Affine Circuits) and \JWcode{RAE} (Randomized Affine Encodings)
which are the same as their \emph{dual} counterparts but split into the
components. So while running the program two \JWcode{RAE}s are derived from one
\JWcode{DRAE}.

As in chapter \ref{def:OPE}, the oblivious polynomial evaluation consists of
three parties: The tamper--proof hardware issuing party \JWpOne{} (executable
program called \JWBpOne{}), the receiving party \JWpTwo{} (\JWBpTwo{}) and the
tamper--proof hardware \JWtoken{} (\JWBtoken{}). The individual programs talk to
each other using TCP/IP networking. The \JWtoken{} implements the OAFE
functionality \JWfuncSym{seq-ot}{OAFE} as provided by the David \& Goliath
protocol \cite{davidgoliath}. The \JWtoken{} does not actually implement the
David \& Goliath protocol but only it's interface. A real implementation using a
tamper--proof hardware token is left open to potential future works.


%
% COMMUNICATION CHANNELS
%
\JWltwo{Communication Channels and Their TCP Ports}
\label{sec:communication-channels}


% GOLIATH TO TOKEN
\JWlthree{\JWpOne{} To \JWtoken{}}
\label{sec:comm:goliath2Token}

On TCP port \JWport{23120} \JWpOne{} initiates a connection to the \JWtoken{}.
This connection is used to transfer the OAFE configuration. After having
successfully received the OAFE configuration, the \JWtoken{} accepts OAFE
evaluation requests (usually from \JWpTwo{}) but no more OAFE configurations.
One \JWtoken{} process can be committed to a OAFE configuration exactly once.


% DAVID TO TOKEN
\JWlthree{\JWpTwo{} To \JWtoken{}}
\label{sec:comm:david2token}

On TCP port \JWport{23021} \JWpTwo{} initiates a connection to the \JWtoken{}.
This connection is used to evaluate the OAFEs. \JWpTwo{} sends a tells which
OAFE to evaluate and the value, the \JWtoken{} then replies with a vector
consisting of the evaluated linear expressions.


% DAVID TO GOLIATH
\JWlthree{\JWpTwo{} To \JWpOne{}}
\label{sec:comm:david2goliath}

On TCP port \JWport{23201} \JWpTwo{} initiates a connection to \JWpOne{}. This
connection is not necessary for the protocol itself but is used to exchange
settings between \JWpOne{} and \JWpTwo{}. \JWpOne{} tells \JWpTwo{} the host
name and the TCP port of the \JWtoken{} and \JWpTwo{} tells \JWpOne{} the port
on which \JWpTwo{} accepts the DRAC.


% GOLIATH TO DAVID
\JWlthree{\JWpOne{} To \JWpTwo{}}
\label{sec:comm:goliath2david}

On TCP port \JWport{23102} \JWpOne{} initiates a connection to \JWpTwo{}. This
connection streams the DRAC that \JWpTwo{} will evaluate.


%
% DIFFERENCES BETWEEN IMPLEMENTATION AND WRITING
%
\JWltwo{Differences Between the Implementation and this Writing}
\label{sec:implementation-differences}

The main difference between the current implementation and this writing is that
the OAFE functionality is not implemented using the David \& Goliath
\cite{davidgoliath} protocol. Instead, the \JWBtoken{} implements a naive
implementation of the ideal OAFE functionality. Further works could change the
implementation to a real implementation of the protocol. Another difference is
that \JWpTwo{} does not verify the polynomial degree it evaluates.


%
% IMPLEMENTATION DETAILS
%
\JWltwo{Implementation Details}
\label{sec:implementation-details}

% Calculations in Finite Fields
\JWlthree{Calculations in Finite Fields}
\label{sec:finite-field-calcs}

For the calculation in the finite fields (chapter \ref{sec:field}) the
\JWTcpp{} library \JWTLntl{} (\emph{Number Theory Library}) has been interfaced
to \JWThaskell{} for the purpose of developing the implementation of this
thesis. The interface has been developed using \JWThaskell{}'s \JWdef{Foreign
Function Interface}{FFI} \cite{haskell2010} and \JWdef{C $\longrightarrow$
Haskell}{C2Hs} \cite{c2hs}. Prior to interfacing the library to Haskell a very
lightweight \JWTc{} wrapper has been developed because the library is written in
\JWTcpp{} which cannot directly be interfaced using the FFI.

Although \JWTntl{} does its work fast and without problems, the library has some
implementation problems and is not optimally designed for exposing its
functionality to functional programming languages. First of all, the library is
neither thread--safe nor reentrant because it globally stores internal state
that can be modified using library functions. The small \JWTc{} wrapper library
therefore tries to provide a thread--safe interface to \JWTntl{} by limiting its
functionality.  Another problem is that \JWTntl{} is designed for mutable
objects, e.g.\ arithmetic operations are implemented using destructive
assignment.  Since \JWThaskell{} is a pure functional language
\cite{haskell2010} and destructive assignment is referentially opaque the
\JWTc{} wrapper changes that and offers only an interface that obeys referential
transparency. It does so by internally allocating a new element for the result
of arithmetic operations and other functions that would modify existing objects
otherwise.

The interface that obeys referential transparency makes the binding work well
inside \JWThaskell{} programs. But for large number of \JWTntl{} objects, there
is a performance problem: creating and destructing a massive amount of objects
is slow. The problem is that it is not possible to allocate memory for \JWTntl{}
objects completely externally because \JWTntl{} allocates memory internally on
the heap in the constructor. In more detail: The objects the programmer gets
from the \JWTntl{} library are only one machine word wide that stores a pointer
to memory on the heap that stores the representation of the elements in the
finite field. In other words: Allocating memory for an array of field elements
is not in $O(1)$ as usual (a chunk of stack space or one \JWcode{malloc} call)
but in $O(n)$ because \JWTntl{} allocates additional memory for every element on
the heap internally. Obviously, the same problem occurs when destructing an
array of \JWTntl{} objects: Usually that is in $O(1)$ as well (waiting until the
stack frame gets destroyed or one \JWcode{free()} call) but using \JWTntl{} this
is in $O(n)$ because \JWTcpp{}'s \JWcode{delete} has to be called for every
element. Not calling \JWcode{delete} will lead to a massive memory leak.
Therefore, the \JWThaskell{} binding cannot manage memory as usual but has to
call a special memory freeing function of the \JWTc{} wrapper that calls
\JWTcpp{}'s \JWcode{delete} for every object. This becomes a major performance
problem for evaluations of large polynomials (see chapter
\ref{sec:comp-complexity}). Nevertheless, this problem is fixable easily by
using a \JWThaskell{} library for finite field calculations or interfacing a
library that fits better.


% Thread Safety
\JWlthree{Thread Safety}

The implementation itself is fully thread--safe, all communication is done using
\JWTghc{}'s implementation of \emph{Software Transactional Memory} (STM)
\cite{stm05} for Haskell. The library (\JWTntl{}) that is used for the
calculations in the finite fields is not thread--safe but the small \JWTc{}
wrapper (see chapter \ref{sec:finite-field-calcs}) tries to fix \JWTntl{}'s
thread--safety problems. However, for guaranteed thread--safety \JWTntl{} should
be replaced by another library that is thread--safe. The \JWTntl{}
implementation also has other disadvantages that would make another library even
more useful (for details see chapter \ref{sec:finite-field-calcs}).


% Testing Correctness
\JWlthree{Testing the Correctness of the Implementation}

The correctness of the implementation has not been proofed formally since that
is not feasible for complex programs. Instead of a real proof, the
implementation has been tested using the specification--driven automatic
\JWThaskell{} testing tool \JWTquickcheck{} \cite{quickcheck} and the classic
unit test approach of \JWTLhunit{}.


% Randomness
\JWlthree{Cryptographic Randomness}

The random numbers needed for the implementation of this thesis are generated
using \JWTLmonadcryptorandom{}. This package allows explicit failure of the
random number generation and exchangeable random number generators such as
\JWTLdrbg{} that implements a NIST standardized number--theoretically secure
random number generator.


% Communication Layer
\JWlthree{Communication Layer}

The communication between the different binaries is done using the schrieblesque
\JWTXLprotobuf{}. This library easily allows network communication with complex
data types independent of the programming language, operating system, machine
endianness, and other parameters.


% Other Libraries
\JWlthree{Other Libraries}

This thesis uses a lot of other libraries mainly from \JWTLhackage{}. The
complete list of packages can be extracted from the file
\JWpath{diplomarbeit.cabal} that comes with the source code of this thesis.


% Source Tree Organization
\JWlthree{Source Tree Organization}
\label{sec:src-org}

A listing of the most important folders in the source code tree of this thesis
can be found below.

\begin{itemize}

  \item \JWpath{programs/}, the source code for the binaries (\JWBpOne{},
    \JWBpTwo{} and \JWBtoken{}).

  \item \JWpath{lib/}, the source code for everything that's reusable.

  \item \JWpath{lib/Functionality/}, the reusable code that forms the binaries'
    functionality.

  \item \JWpath{lib/Data/OAFE/}, OAFE functionality implementation.

  \item \JWpath{lib/Data/RAE/}, RAE encoding, decoding and evaluation.

  \item \JWpath{lib/Math/}, the finite field implementation and other
    mathematical methods.

  \item \JWpath{test/}, the \JWTquickcheck{} tests and the unit tests.

  \item \JWpath{bench/}, benchmarking programs.

  \item \JWpath{scripts/}, helper scripts, mainly for benchmarking.

  \item \JWpath{protos/}, the \JWTXprotobuf{} definition files.

  \item \JWpath{gen-src/}, automatically generated source code, derived from the
    \JWTprotobuf{} definition files.

  \item \JWpath{writings/}, this writing.

  \item \JWpath{evaluation/}, data for the figures of this writing.

  \item \JWpath{dist/}, after having compiled the source code, this folder
    contains the binaries.

\end{itemize}


% Build System
\JWlthree{Build System and Building}

The implementation can be easily built using \JWTLcabal{} by typing:

\begin{lstlisting}

cd diplomarbeit/
cabal install --only-dependencies
cabal configure
cabal build

\end{lstlisting}


%
% USER INPUT/OUTPUT
%
\JWltwo{User Input/Output}
\label{sec:user-io}

Whenever field elements in large fields that are no prime fields (e.g.\ %
$\mathbb{F}_{2^{256}}$) are read from user input or are outputted to the user,
the following format (given as a regular expression) is used:

\JWcode{\textbackslash[([01]( [01])\{0,255\})?\textbackslash]}

Intuitively that is between $0$ and $256$ digits (each \JWcode{0} or \JWcode{1})
separated by spaces and surrounded by square brackets. The meaning of such a
string is $\sum_{i=1}^P d \cdot x^p$ where $p$ is the position of the digit in
the string and $P$ is the maximal position (counted from $0$). And exception is
the string \JWcode{[]} which represents the polynomial \JWcode{0}. Examples:

\begin{itemize}

  \item \JWcode{[]} and \JWcode{[0]} represent $0$

  \item \JWcode{[1]} represents $1 \cdot x^0 = 1 \cdot 1 = 1$

  \item \JWcode{[0 1]} represents $0 + 1 \cdot x^1 = x$

  \item \JWcode{[1 0]} represents $1 + 0 \cdot x^1 = 1$

  \item \JWcode{[1 1]} represents $1 + 1 \cdot x^1 = 1 + x$

  \item \JWcode{[1 0 1 0 1 0]} represents $1 + x^2 + x^4$

\end{itemize}

The default implementation uses $\mathbb{F}_{2^{256}}$ specified by the
irreducible polynomial $1 + x^2 + x^5 + x^{10} + x^{256}$. The $\mathbb{F}_{97}$
implementation uses regular digits to read and write the field elements because
it is a prime field. Examples: $1 \hat{=} 1 (mod 97) = 1_{\mathbb{F}_{97}}$, $98
\hat{=} 98 (mod 97) = 1_{\mathbb{F}_{97}}$.


%
% USAGE OF THE PROGRAMS
%
\JWltwo{Usage of the Programs}
\label{sec:usage}

The three main programs are \JWBpOne{}, \JWBpTwo{} and \JWBtoken{}. \JWBtoken{}
does not accept any command line parameters. \JWpOne{} expects exactly one
command line parameter: The file for the polynomial to evaluate, one coefficient
per line. \JWpTwo{} expects exactly one parameter, too: The field element to
evaluate the polynomial. Example:

\JWcmd{Goliath /tmp/my-polynomial.txt}

\JWcmd{Token}

\JWcmd{David '[1 0 1 1 1 0 1]'}


%
% DOCUMENTATION
%
\JWltwo{Documentation}
\label{sec:implementation-doc}

The implementation is documented using \JWTLhaddock{}.

\JWtodo{Add link to Haddock doc}


%
% CODE AVAILABILITY
%
\JWltwo{Code Availability}
\label{sec:code-availability}

All of the code is open--sourced and available at
\JWnamedlink{https://github.com/weissi/diplomarbeit.git}{GitHub}. The code can
be obtained by typing:

\begin{lstlisting}
git clone https://github.com/weissi/diplomarbeit.git
\end{lstlisting}


% vim: set spell spelllang=en_us fileencoding=utf8 :
