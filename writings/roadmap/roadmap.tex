\documentclass[12pt, a4paper]{article}

% language
\usepackage[american]{babel}
\usepackage[utf8]{inputenc}

% bib
\bibliographystyle{plainurl}

\usepackage{amsmath}
\usepackage{amsfonts}
\usepackage{url}

% I need them :-)
\usepackage[plainpages=false]{hyperref}
\hypersetup {
   pdfauthor={Johannes Wei\ss},
   pdftitle={Dimplomarbeit -- Roadmap},
   pdfsubject={Diploma Thesis Roadmap},
   pdfkeywords={KIT, crypto, diploma thesis, roadmap}
}
\usepackage{listings}
\usepackage{color}

\title{Diplomarbeit -- Roadmap}

\begin{document}

\maketitle

This writing describes the process of securely evaluating arithmetic functions
such as $f(x_G,x_D) = ...$ over fields. The main building block is the David \&
Goliath OAFE protocol\cite{davidgoliath}.


\section{From Arithmetic Formulas To Matrix Multiplications}
\label{sec:FormulasToMatrixMuls}

Our definition of formulas is the same as in \cite{cleve91}: Formulas are
circuits that are trees. A postorder traversal is enough to evaluate the formula
easily. We describe the evaluation of such a formula using \emph{linear
bijection straight--line programs} (LBS programs)\cite{cleve91} which use at
most $\omega$ registers. A LBS program can be simulated by matrix
multiplications, one statement is simulated by one matrix multiplication. The
matrices are elements of $SL_w(K)$, the special linear group consisting of
$\omega \times \omega$ matrices with determinant $1$ (and $K$ a field).

A LBS program consists of assignment statements of the following
forms where $R_{1,...,\omega}$ denote registers, $c \in K$ constants and $x_u
\in K$ the formula's inputs:

\begin{align}
R_j & \leftarrow R_j + (R_i \cdot c) \\
R_j & \leftarrow R_j - (R_i \cdot c) \\
R_j & \leftarrow R_j + (R_i \cdot x_u) \\
R_j & \leftarrow R_j - (R_i \cdot x_u)
\end{align}


\subsection{Transformation of Formulas to LBS Programs}

The goal is to transform a register $R_{out}$ with a initial value of $0$ to be
transformed like $R_{out} \leftarrow R_{out} + R_{one} \cdot f(x_G,x_D)$ . The
special register $R_{one}$ holds a constant $1$. This can be achieved by
induction as follows.  For the exact definitions, proofs and algorithms how to
transform arbitrary formulas to LBS programs see \cite{cleve91}.


\paragraph{Depth $d = 0$}

The construction of the LBS for $d = 0$ is very straightforward:
$R_j \leftarrow R_j \pm R_i \cdot c$ or $R_j \leftarrow R_j \pm R_i \cdot x_u$ .


\paragraph{Depth $d > 0$}

Having LBS progams that do $R_j \leftarrow R_j \pm R_i \cdot l(x_G, x_D)$  and
$R_j \leftarrow R_j \pm R_i \cdot r(x_G, x_D)$ we can
transform formulas of depth $d > 0$ to a LBS program using only formulas of
of depth $d - 1$ until $d = 0$.

\subparagraph{Additive:} We can construct a LBS program doing $R_j \leftarrow
R_j + R_i \cdot (l + r)(x_G, x_D)$ easily by the following LBS program

\begin{align*}
R_j & \leftarrow R_j + R_i \cdot l(x_G, x_D) \\
R_j & \leftarrow R_j + R_i \cdot r(x_G, x_D)
\end{align*}

Alike for $R_j \leftarrow R_j - R_i \cdot (l + r)(x_G, x_D)$

\begin{align*}
R_j & \leftarrow R_j - R_i \cdot l(x_G, x_D) \\
R_j & \leftarrow R_j - R_i \cdot r(x_G, x_D)
\end{align*}


\subparagraph{Multiplicative:} We can construct a LBS program doing $R_j
\leftarrow R_j + R_i \cdot (l \cdot r)(x_G, x_D)$ less obviously by the LBS program

\begin{align*}
R_k & \leftarrow R_k - R_j \cdot r(x_G, x_D) \\
R_j & \leftarrow R_j + R_i \cdot l(x_G, x_D) \\
R_k & \leftarrow R_k + R_j \cdot r(x_G, x_D) \\
R_j & \leftarrow R_j - R_i \cdot l(x_G, x_D)
\end{align*}

Alike for $R_j \leftarrow R_j - R_i \cdot (l \cdot r)(x_G, x_D)$

\begin{align*}
R_k & \leftarrow R_k - R_j \cdot r(x_G, x_D) \\
R_j & \leftarrow R_j - R_i \cdot l(x_G, x_D) \\
R_k & \leftarrow R_k + R_j \cdot r(x_G, x_D) \\
R_j & \leftarrow R_j + R_i \cdot l(x_G, x_D)
\end{align*}


\subsection{Current Implementation State}

The current implementation is a Haskell library which does the transformation
from the formula, over the LBS program to the matrices. Exemplary,
the definition of the function $f(x_G,x_D) = 3x \cdot (x_G + x_D^2)$ looks like

\lstset{language=Haskell}

\begin{lstlisting}
_Xg_ :: Expr
_Xg_ = Var "Xg"

_Xd_ :: Expr
_Xd_ = Var "Xd"

f :: Expr
f = 3 * _Xg_ * (_Xg_ + _Yd_ * _Yd_)
\end{lstlisting}

\noindent{}The resulting LBS program which will hold the result in \texttt{R1}
looks like:

\begin{lstlisting}
R1 <- R1 - R2 * Xg
R1 <- R1 - R3 * Xd
R3 <- R3 - R2 * Xd
R1 <- R1 + R3 * Xd
R3 <- R3 + R2 * Xd
R2 <- R2 - R3 * Xg
R3 <- R3 + R0 * 3
R2 <- R2 + R3 * Xg
R3 <- R3 - R0 * 3
R1 <- R1 + R2 * Xg
R1 <- R1 - R3 * Xd
R3 <- R3 + R2 * Xd
R1 <- R1 + R3 * Xd
R3 <- R3 - R2 * Xd
R2 <- R2 - R3 * Xg
R3 <- R3 - R0 * 3
R2 <- R2 + R3 * Xg
R3 <- R3 + R0 * 3
\end{lstlisting}

\noindent{}The construction of the matrices is straight--forward: The statement
$R_i \leftarrow R_i + (R_j \cdot \alpha)$ is equivalent to the $K^{\omega \times
\omega}$ identity matrix whose entry $i,j$ is set to $\alpha$.


\section{Grouping the Matrices}
\label{sec:matrix-grouping}

The grouping process is very straightforward, too: From the process described in
section \ref{sec:FormulasToMatrixMuls} we obtain matrices $\widehat{M_1}$ to
$\widehat{M_n}$ which each have the effect of exactly one LBS program statement.
Using associativity, we group together a variable amount of matrices
$\widehat{M_1}$ to $\widehat{M_n}$. Each group is complete when there is at least
one reference to the \emph{other party's input} $x_D$.  Matrices $M_1$ to $M_m$
where $M_1$ to $M_{m-1}$ definitely have at least one reference to $x_D$ and
$M_m$ may or may not are the result of this step.  Obviously the following
properties hold:

\begin{align*}
n & \geq m \\
\prod_{i=1}^m M_i & = \prod_{j=1}^n \widehat{M_j}
\end{align*}

\section{Garbling the Matrices}
\label{sec:matrix-garbling}

Let $D_L$ be the $\omega \times \omega$ matrix whose entry $2,2$ is $1$, and
whose other entries are $0$. Multiplication of $D_L$ selects the second row of
matrices multiplied on the right of $D_L$. Let $D_R$ be the $\omega \times
\omega$ matrix whose entry $1,1$ is $1$, and whose other entries are $0$. This
matrix will select the first column when multiplied on the left of any matrix.
Using additional matrices $S_1$ to $S_{m}$ uniformly at random and invertible,
we can build up $m$ garbled matrix groups:

\begin{align*}
U_1 & = D_L M_1 S_1 \\
U_i & = S_{i-1}^{-1} M_i S_i &
\text{for $i \in \{n \in \mathbb{N} \big| 1 < n < m\}$}\\
U_m & = S_{m-1}^{-1} M_m D_R
\end{align*}

\noindent{} Hence, each $U_{1..m}$ does not reveal usable information by itself
\cite{cramer03}, but $\prod_{i=1}^m U_i$ does still calculate the desired
result.


\section{Evaluating them using OAFEs (David \& Goliath)}

From section \ref{sec:matrix-garbling} we obtain the matrices $U_{1..m} \in
K^{\omega \times \omega}$. We can easily reshape the matrices $U_{1..m}$ to
vectors $u_{1..m} \in K^{\omega^2}$ and concatenate the vectors $u_{1..m}$ to
one giant vector $\mu \in K^{m\omega^2}$. After that we deduce two vectors $a$
and $b$ that hold the following property ($a, b \in K^{m\omega^2}$, $x_D$ a
scalar variable, as in section \ref{sec:matrix-grouping} David's input):

\begin{align}
a \cdot x_D + b = \mu
\end{align}

Using the $\prod^{\text{semi-int}}_{\text{OAFE}}$ protocol\cite{davidgoliath} we
are now ready to evaluate the function securely:

\begin{itemize}

\item The setup is: $a, b$ as above, $\mathbb{F}_q = K$, $k = m\omega^2$ and $m$
known to both, David and Goliath

\item After applying the protocol, David is now able to evaluate the linear
functions to his result vector $y \in K^{m\omega^2} = GWh + \tilde{a}x_D +
\tilde{b}$ ($G$, $W$, $h$, $\tilde{a}$ ,$\tilde{b}$ as in the paper
\cite{davidgoliath})

\item The last but one step is to reshape $y$ to the matrices $F_{1..m}
\in K^{m\omega^2}$

\item Finally, the entry $2, 1$ of $\prod_{i=1}^m F_i$ is the desired result of
$f(x_G,x_D)$.

\end{itemize}

\addcontentsline{toc}{chapter}{Bibliography}
\bibliography{bibliography}

\end{document}
% vim: set spell spelllang=en_us fileencoding=utf8 :
